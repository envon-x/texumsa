
\newglossaryentry{AcondicionamientoAire}{
        name={Acondicionamiento del aire},
        description={Proceso de tratatamiento del aire para cumplir con los requisitos de un espacio acondicionado mediante el control de la temperatura, humedad, limpieza y distribución del aire %\parencite[ p. 3]{Hedrick2013}
    }
}


\newglossaryentry{Climatol}{
name={Climatología},
description={}
}

\newglossaryentry{Clima12}{
    name={Clima},
    description={    % Es la síntesis de 30 a\~nos o más, de los registros meteorológicos.
        Cuando afirmamos que en un lugar específico (por ejemplo, nuestra ciudad), los inviernos son secos y frios; realmente estamos haciendo una valoración de las condiciones meteorológicas de muchos años; intuitivamente estamos definiendo el clima \parencite[]{Jimenez2004} como una coondición permamente y propia del lugar. De esta forma definimos \textit{clima}, como la síntesis de 30 a\~nos o más, de las condiciones meteorológicas \parencite[]{Tejeda-Martinez2018}. A menudo el término clima se confunde con las condiciones meteorológicas o tiempo atmosférico cuando realmente se debería llamarse así, por su nombre: condiciones meteorológicas, tiempo atmosférico o simplemente, tiempo.}
}

\newglossaryentry{Climatiz1}{
    name={Climatatización},
    description={Término utilizado de manera inadecuada en los procesos \acrshort{hvac}; que se refiere a acondicionar el aire (Air Conditioning, en Inglés). El término apropiado para describir este proceso es, \textbf{acondicionamiento del aire}, y lo que busca es confort (bienestar o comodidad material, de acuerdo a la \acrshort{rae}), para un momento determinado y específico; es decir una condición no permanente. Mientras que climatización, palabra derivada de \textit{clima}, se refiere a una forma de acondicionmiento del aire casi inalterable en el tiempo, no menos de 30 años. En los procesos \acrshort{hvac}, este último no sucede.}
}

 \newglossaryentry{Entalp}{
    name={Entalpía},
    description={   Se define como la suma de la energía interna %($\symInternalEnergy$) 
        y el flujo de trabajo %($\symPressure \symVolume$)
        \parencite[p. 137]{Gatley2013}. El profesor R. Mollier, famoso por su diagrama, se refirió a la \textit{entalpía} como contenido de calor y calor total \parencite[p. 125]{YunusA.Cengel2015}.}
 }


 \newglossaryentry{EntalpEspcific}{
 name={Entalpía Específica},
 description={   % \soutthick{Indica el total de energía térmica que contiene la mezcla. puede expresarse en la}
    % fracción de los componentes que hacen la mezcla.}
    Se define como la entalpía respecto a una unidad de masa; es una propiedad intensiva.}
}
    
\newglossaryentry{HumedadEspecific}{
name={Humedad Específica},
description={    Es la relación de la masa de vapor y la masa del gas seco.}
}

\newglossaryentry{HumedadRelativ}{
name={Humedad Relativa},
description={    Se define como la relación entre la persión parcial del vapor en la mezcla gaseosa  y la presión de vapor, expresada en tanto por uno o en tanto por ciento. }
}

% % \section{Huella Ambiental}
% % \section{Huella de Carbono}
% % \section{Impacto Ambiental}
% \section{Impacto Ambiental}

\newglossaryentry{Meteorolog}{
    name={Meteorología},
    description={    Es la ciencia que se encarga de estudiar la atmosféra, sus propiedades y los fenómenos que ocurren en ella \parencite[]{Jimenez2004}; tales como, la lluvia, nevada, etc. A este conjunto de fenómenos se llama, meteoros. Los comienzos de  meteorología fue en base a observaciones empíricas. Hoy en día, es una ciencia basada en los conocimientos de la Física y las Matemáticas.}
}   





\newglossaryentry{TempeAparente}{
name={Temperatura aparente},
description={    % (Tejada, 2018) }
    Propuesta por \href{https://es.planetcalc.com/2089/}{R. Steadman (1979)}; se define como la temperatura a la cual una combinación dada de temperatura de bulbo seco y humedad relativa una persona percibe el mismo \parencite[p. 220]{Tejeda-Martinez2018}. }
}

\newglossaryentry{TempBulbSeco1}{
name={Temperatura de Bulbo Seco},
description={Para \parencite[]{greenwade93}, es la temperatura del aire medido por un termómetro o sensor, expuesto al aire atmosférico y libre de la exposición directa al sol.}
}    
% \newglossaryentry{Punto}{
% name={Punto de Burbuja},
% description={    }
% \newglossaryentry{Punto}{
% name={Punto de Rocío},
% description={        % Se define como la temperatura debajo del cual el vapor en el gas, comienza a condensarse. %También es el punto de 100 \% de saturación}
    
\newglossaryentry{PuntoRocio}{
name={Punto de Rocío},
description={Es la temperatura debajo del cual el vapor en el gas, comienza a condensarse. Sucede cuando la temperatura de bulbo seco y húmedo alcanzan el mismo valor; es decir, cuando el gas alcanza las condiciones de saturación y empieza a aparecer la primera gota de rocío. % Temperatura cuando el vapor de agua alcanza el punto de saturación (100 \% HR)}}        
}}
 
% \newglossaryentry{SistemDef}{
% name={Sistema}
% description={un sistema termodinámico es aquella parte del universo físico cuyas propiedades se están investigando; generalmente está confinado a un lugar definido en el espacio por una \textit{frontera} que lo separa del resto de universo, el entorno (Castellan, 1987).}
% }


%     \newglossaryentry{PropSist}{
% name={Propiedades del sistema}
% description={Son aquellos atributos físicos que se perciben por los sentidos o por ciertos métodos experimentales de investigación; pueden ser: (1) medibles y (2) no medibles (Castellan, 1987).}
% }

% \newglossaryentry{EstadoSistema}{
% name={Estado de un sistema}
% description={El estado de un sistema se encuentra definido cuando cada una de sus propiedades tiene un valor determinado; el cual se consigue mediante un estudio experimental del sistema o a través de experiencias con sistemas sejemantes (Castellan, 1987).}
% }

\newglossaryentry{TempeBulboHumed}{
name={Temperatura de Bulbo Húmedo},
description={Es la temperatura indicada por el termómetro de bulbo seco pero, con el bulbo cubierto o bañado de humedad. Su valor es igual o menor a la termperatura de bulbo seco.}
}
    
\newglossaryentry{TiempoAtmsferico}{
name={Tiempo atmosférico},
description={    Tiempo atmosférico ó simplemente tiempo (weather, en Inglés). Es la descripción de las condiciones atmosféricas en un momento y lugar dado. Reporte que dan los medios de comunicación a diario \parencite[p. 6]{Jimenez2004}. %Principal condicionante de las actividades del hombre.}
}
}

% \newglossaryentry{Siste}{
% name={Sistemas climáticos},
% description={}

\newglossaryentry{Ventilacion}{
name={Ventilación} ,
description={    Proceso que añade o remueve aire de manera natural o mecánica desde o hacia un espacio; el aire puede ser acondicionado o no (ASHRAE, 2012).}
}

\newglossaryentry{VolumEspecif}{
name={Volumen Especifico},
description={Representa el volumen que ocupa un kilogramo de aire seco en una condición dada \parencite{greenwade93}.}
}
