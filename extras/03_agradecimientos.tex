
\chapter{Agradecimientos}

\cabeceraEspecial{Agradecimientos}

\begin{FraseCelebre}
\begin{Frase}
A todos los que la presente vieren y entendieren.
\end{Frase}
\begin{Fuente}
Inicio de las Leyes Orgánicas. Juan Carlos I
\end{Fuente}
\end{FraseCelebre}

Groucho Marx decía que encontraba a la televisión muy educativa porque
cada vez que alguien la encendía, él se iba a otra habitación a leer
un libro. Utilizando un esquema similar, nosotros queremos agradecer
al Word de Microsoft el habernos forzado a utilizar \LaTeX. Cualquiera
que haya intentado escribir un documento de más de 150 páginas con
esta aplicación entenderá a qué nos referimos. Y lo decimos porque
nuestra andadura con \LaTeX\ comenzó, precisamente, después de
escribir un documento de algo más de 200 páginas. Una vez terminado
decidimos que nunca más pasaríamos por ahí. Y entonces caímos en
\LaTeX.

Es muy posible que hubíeramos llegado al mismo sitio de todas formas,
ya que en el mundo académico a la hora de escribir artículos y
contribuciones a congresos lo más extendido es \LaTeX. Sin embargo,
también es cierto que cuando intentas escribir un documento grande
en \LaTeX\ por tu cuenta y riesgo sin un enlace del tipo ``\emph{Author
  instructions}'', se hace cuesta arriba, pues uno no sabe por donde
empezar.

Y ahí es donde debemos agradecer tanto a Pablo Gervás como a Miguel
Palomino su ayuda. El primero nos ofreció el código fuente de una
programación docente que había hecho unos años atrás y que nos sirvió
de inspiración (por ejemplo, el fichero \texttt{guionado.tex} de
\texis\ tiene una estructura casi exacta a la suya e incluso puede
que el nombre sea el mismo). El segundo nos dejó husmear en el código
fuente de su propia tesis donde, además de otras cosas más
interesantes pero menos curiosas, descubrimos que aún hay gente que
escribe los acentos españoles con el \verb+\'{\i}+.

No podemos tampoco olvidar a los numerosos autores de los libros y
tutoriales de \LaTeX\ que no sólo permiten descargar esos manuales sin
coste adicional, sino que también dejan disponible el código fuente.
Estamos pensando en Tobias Oetiker, Hubert Partl, Irene Hyna y
Elisabeth Schlegl, autores del famoso ``The Not So Short Introduction
to \LaTeXe'' y en Tomás Bautista, autor de la traducción al español. De
ellos es, entre otras muchas cosas, el entorno \texttt{example}
utilizado en algunos momentos en este manual.

También estamos en deuda con Joaquín Ataz López, autor del libro
``Creación de ficheros \LaTeX\ con {GNU} Emacs''. Gracias a él dejamos
de lado a WinEdt y a Kile, los editores que por entonces utilizábamos
en entornos Windows y Linux respectivamente, y nos pasamos a emacs. El
tiempo de escritura que nos ahorramos por no mover las manos del
teclado para desplazar el cursor o por no tener que escribir
\verb+\emph+ una y otra vez se lo debemos a él; nuestro ocio y vida
social se lo agradecen.

Por último, gracias a toda esa gente creadora de manuales, tutoriales,
documentación de paquetes o respuestas en foros que hemos utilizado y
seguiremos utilizando en nuestro quehacer como usuarios de
\LaTeX. Sabéis un montón.

Y para terminar, a Donal Knuth, Leslie Lamport y todos los que hacen y
han hecho posible que hoy puedas estar leyendo estas líneas.

\endinput
% Variable local para emacs, para  que encuentre el fichero maestro de
% compilación y funcionen mejor algunas teclas rápidas de AucTeX
%%%
%%% Local Variables:
%%% mode: latex
%%% TeX-master: "../Tesis.tex"
%%% End: