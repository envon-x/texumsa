%---------------------------------------------------------------------
%
%                            TeXiS_cover.tex
%
%---------------------------------------------------------------------
%
% TeXiS_cover.tex
% Copyright 2009 Marco Antonio Gomez-Martin, Pedro Pablo Gomez-Martin
%
% This file belongs to TeXiS, a LaTeX template for writting
% Thesis and other documents. The complete last TeXiS package can
% be obtained from http://gaia.fdi.ucm.es/projects/texis/
%
% This work may be distributed and/or modified under the
% conditions of the LaTeX Project Public License, either version 1.3
% of this license or (at your option) any later version.
% The latest version of this license is in
%   http://www.latex-project.org/lppl.txt
% and version 1.3 or later is part of all distributions of LaTeX
% version 2005/12/01 or later.
%
% This work has the LPPL maintenance status `maintained'.
% 
% The Current Maintainers of this work are Marco Antonio Gomez-Martin
% and Pedro Pablo Gomez-Martin
%
%---------------------------------------------------------------------
%
% Fichero que  contiene la portada y  primera hoja de la  tesis, en la
% que se vuelve a repetir el t\'itulo.
%
% El contenido  de la  portada se configura  en un fichero  externo en
% extras/cover.tex
%
%---------------------------------------------------------------------

%%%
% Gesti\'on de la configuraci\'on
%%%

% T\'itulo de la portada
\def\tituloPortadaVal{\titulo}
\newcommand{\tituloInvestigacion}[1]{
  \def\tituloPortadaVal{#1}
}

\def\tituloAOptarVal{\gradoAcademico} % variable: tituloAOptarVal: para  usar denro de la macro
\newcommand{\tituloAcademico}[1]{     % tituloAcademico: para introducir el valor de que se desea asignar a tituloAOptarVal.
  \def\tituloAOptarVal{#1}
}

\def\nombreAsesorVal{\asesor}
\newcommand{\asesorInvestigacion}[1]{
  \def\nombreAsesorVal{#1}
}

% Imagen de la portada (escudo)
\def\imagenPortadaVal{resources/images/vectorial/todo}
\newcommand{\imagenPortada}[1]{
  \def\imagenPortadaVal{#1}
}

\def\escalaImagenPortadaVal{1.0}
\newcommand{\escalaImagenPortada}[1]{
  \def\escalaImagenPortadaVal{#1}
}

% Autor
\def\autorPortadaVal{\autor}
\newcommand{\autorPortada}[1]{
  \def\autorPortadaVal{#1}
}

% Tutor
\def\directorPortadaVal{Tutor no definido. Usa
  \texttt{$\backslash$tutorPortada}}
\newcommand{\directorPortada}[1]{
  \def\directorPortadaVal{#1}
}

% Lugar de publicacion
\def\lugarPublicacionVal{\lugar}
\newcommand{\lugarPublicacion}[1]{
  \def\lugarPublicacionVal{#1}
}

% Fecha de publicacion
\def\fechaPublicacionVal{}
\newcommand{\fechaPublicacion}[1]{
  \def\fechaPublicacionVal{#1}
}

% Tipo de documento (TESIS, MANUAL, ...)
\def\tipoDocumentoVal{TESIS DOCTORAL}
\newcommand{\tipoDocumento}[1]{
  \def\tipoDocumentoVal{#1}
}

% Institución
\def\institucionVal{}
\newcommand{\institucion}[1]{
  \def\institucionVal{#1}
}

% Primer subt\'itulo de la segunda portada
\def\textoPrimerSubtituloPortadaVal{%
\textit{Memoria que presenta para optar al t\'itulo de: Ingeniero Ambiental}  \\ [0.3em]%
  \textbf{\autorPortadaVal}%
}
\newcommand{\textoPrimerSubtituloPortada}[1]{
  \def\textoPrimerSubtituloPortadaVal{#1}
}

% Segundo subt\'itulo de la segunda portada
\def\textoSegundoSubtituloPortadaVal{%
\textit{Tutor}  \\ [0.3em]
  \textbf{\directorPortadaVal}
}
\newcommand{\textoSegundoSubtituloPortada}[1]{
  \def\textoSegundoSubtituloPortadaVal{#1}
}

% ISBN
\newcommand{\isbn}[1]{
  \def\isbnVal{#1}
}

% Copyright
\newcommand{\copyrightInfo}[1]{
  \def\copyrightInfoVal{#1}
}

% Cr\'editos a TeXiS
\newcommand{\noTeXiSCredits}{
  \def\noTeXiSCreditsVal{}
}

% Explicaci\'on sobre impresi\'on a doble cara
\newcommand{\explicacionDobleCara}{
  \def\explicacionDobleCaraVal{}
}
%%%
% Configuraci\'on terminada
%%%


%%%
%% COMANDO PARA CREAR LAS PORTADAS.
%% CONTIENE TODO EL C\'oDIGO LaTeX
%%%
\newcommand{\makeCover}{

  % Ponemos el marcador en el PDF
  \ifpdf
  \pdfbookmark{P\'agina de T\'itulo}{titulo}
  \fi

  %
  %  M\'aRGENES
  %
  % La maquetaci\'on de las p\'aginas en LaTeX es bastante complicada en lo
  % que se refiere a los m\'argenes. Por ejemplo, _siempre_ hay un
  % desplazamiento de una pulgada hacia la derecha y hacia abajo, porque
  % por razones de maquetaci\'on se cree que es necesario ese espacio. Si
  % quieres (en una hoja impar, que son de las \'unicas que me he
  % preocupado) escribir por encima de la primera pulgada, o hacia la
  % izquierda, tienes que usar "m\'argenes" negativos.
  % 
  % Para poder seguir esto un poco, lo mejor es que ejecutes \layout con
  % el paquete layout cargado, para ver una imagen...
  %
  % Los valores importantes en lo que se refiere al margen (horizontal,
  % que es del \'unico que me he preocupado) son:
  %
  %   - \hoffset : desplazamiento horizontal del "eje" de coordenadas. A
  %   este valor hay que sumarle, irremediablemente, una pulgada. El
  %   valor por defecto es 0 pt
  %   - \oddsidemargin : "margen" izquierdo (en las p\'aginas impares). El
  %   texto principal de los p\'arrafos comenzar\'a en esa posici\'on (es
  %   decir, 1 pulgada + \hoffset + \oddsidemargin). El valor por
  %   defecto es 22 pt
  %   - \textwidth : longitud del texto (de los p\'arrafos). El valor por
  %   defecto son 360 pt.
  %   - \marginparsep : separador de la parte derecha del texto
  %   principal y el espacio a anotaciones en el margen (notas
  %   marginales). El valor por defecto es 7 pt.
  %   - \marginparwidth : ancho de la secci\'on de notas marginales. El
  %   valor por defecto es 106 pt.
  %
  % Fijate que las notas marginales NO necesariamente llegar\'an hasta el
  % final del folio. La separaci\'on entre el extremo derecho de las notas
  % al margen y el final del folio en realidad depender\'a del tamaño de
  % \'este; no se especifica de ninguna manera. Esto significa que si
  % quieres ajustar exactamente el margen derecho tienes que echar
  % cuentas con respecto al tamaño del papel, que se mantiene en
  % \paperwidth.
  %
  % En realidad, no s\'e de qu\'e manera pero todos los valores est\'an
  % relacionados de alg\'un modo, y un cambio en uno afecta a los dem\'as de
  % maneras bastante inveros\'imiles. Adem\'as, _s\'olo_ pueden cambiarse en
  % el pre\'ambulo (bueno... al menos \textwidth, \oddsidemargin se puede
  % cambiar en otros sitios, pero no se pueden hacer muchas cosas s\'olo
  % con aquellos que se pueden cambiar...).
  %
  % En general se recomienda que NO cambies los m\'argenes. Est\'an elegidos
  % por especialistas que saben de maquetaci\'on y que han estudiado en
  % profundidad las mejores organizaciones. Por ejemplo, los cambios que
  % hagas pueden alargar demasiado las l\'ineas, o dejarlas demasiado
  % cortas. Pero bueno, si aun as\'i quieres cambiar los m\'argenes _a nivel
  % global_ es preferible que uses el paquete geometry, cuya inclusi\'on
  % recibe los cent\'imetros de m\'argen que quieres en cada lado, y \'el se
  % encarga de hacer las cuentas para que queden as\'i, porque tocar tanto
  % valor es un infierno.
  %
  % Si quieres cambiar los m\'argenes moment\'aneamente, entonces lo mejor
  % es hacer uso de un entorno tipo "lista" que permite tocar algunos
  % contadores para ajustar las posiciones de los p\'arrafos. Eso es
  % precisamente lo que hace el entorno cambiamargen definido en
  % TeXiS.sty
  %
  % El problema es que esos cambios son _relativos_ a los m\'argenes
  % oficiales. Si quieres hacer un cambio dr\'astico (como el que
  % necesitamos en la portada, para que quede centrada), entonces es
  % necesario echar cuentas con los contadores anteriores para realizar
  % el desplazamiento adecuado en cada lado para conseguirlo.
  %
  % Para resumir, los valores usados en los m\'argenes eran:
  % 1 pulgada + hoffset + oddsidemargin +
  %          + textwidth +
  % + marginparsep + marginparwidth + AJUSTE 
  %                                          = paperwidth
  %
  % Lo que necesitamos en crear un entorno cambiamargen pasando los
  % valores adecuados para que quede centrado. Para eso, hay que hacer
  % cuentas, y eso es un tanto infierno en LaTeX a no ser que se incluya
  % el paquete calc que permite notaci\'on infija de operadores. Por
  % tanto, para que esto funcione hay que incluirlo.
  %
  % Para aclararnos, vamos a crear un par de longitudes para hacer las
  % cuentas poco a poco. Adem\'as, lo primero es asumir que queremos
  % _eliminar todos los m\'argenes_ y que los p\'arrafos lleguen totalmente
  % de lado a lado. Para eso, en la izquierda tendremos que restar la
  % suma de los tres primeros valores (1 pulgada, \hoffset y
  % \oddsidemargin).

  \newlength{\cambioIzquierdo}
  \setlength{\cambioIzquierdo}{1in + \hoffset + \oddsidemargin}
  % Si te falla aqu\'i, incluye el paquete calc en el pre\'ambulo.
  % 1in = 1 pulgada = 72.27 pt

  % En la parte derecha hay que restar el "margen" visible total, que es
  % el tamaño de la hoja restando el espacio hasta la izquierda del
  % p\'arrafo y su ancho. Aprovechamos que el primer espacio lo tenemos en
  % \cambioIzquierdo.
  \newlength{\cambioDerecho}
  \setlength{\cambioDerecho}{\cambioIzquierdo + \textwidth}
  \setlength{\cambioDerecho}{\paperwidth - \cambioDerecho}

  % Ya casi est\'a. Si hicieramos
  %
  % \begin{cambiamargen}{-\cambioIzquierdo}{-\cambioDerecho}
  %    ...
  % \end{cambiamargen}
  %
  % tendriamos p\'arrafos que van de extremo a extremo de la hoja. Como
  % eso es una exageraci\'on, restamos a cada longitud el m\'argen real que
  % queremos dejar.-
  \newlength{\margenPortada}
  \setlength{\margenPortada}{2.54cm}

  \setlength{\cambioIzquierdo}{\cambioIzquierdo - \margenPortada}
  \setlength{\cambioDerecho}{\cambioDerecho - \margenPortada}


  %%%%%%%%%%%%%%%%%%%%%%%%%%%%%%%%%%%%%%%%%%%%%%%%%%%%%%%%%%%%%%%%%%%%%%%%%%%%%%%%%%%%%%%%
  % Portada
  %%%%%%%%%%%%%%%%%%%%%%%%%%%%%%%%%%%%%%%%%%%%%%%%%%%%%%%%%%%%%%%%%%%%%%%%%%%%%%%%%%%%%%%%

  % P\'agina sin cabeceras
  \thispagestyle{empty}

  \begin{cambiamargen}{-\cambioIzquierdo}{-\cambioDerecho}


      % En la primera hoja no se entiende de p\'aginas pares e impares
      \newlength{\evensidemarginOriginal}
      \setlength{\evensidemarginOriginal}{\evensidemargin}

      \newlength{\oddsidemarginOriginal}
      \setlength{\oddsidemarginOriginal}{\oddsidemargin}

      \setlength{\evensidemargin}{0cm}
      \setlength{\oddsidemargin}{0cm}

      \vfill

    %%%%%%%%%%%%%%%%%%%%%%%%%%%%%%%%%%%%%%%%%%% DISEÑO %%%%%%%%%%%%%%%%%%%%%%%%%%%%%%%%%%%%%%%%%%%%%
      \begin{large}
        \begin{center}
          
          \textbf{\institucionVal}%\\[1em]
        \end{center}
      \end{large}

      \vfill

      \begin{center}
        \vfill
        \includegraphics[scale=\escalaImagenPortadaVal]{\imagenPortadaVal}\\[5mm]
        
      \end{center}

      \begin{Huge}
        \begin{center}
          {\textbf{\tipoDocumentoVal}}
        \end{center}
      \end{Huge}

      \vfill

    \begin{large}
      \begin{center}
        % \vskip 1cm
        \textbf{\tituloPortadaVal}
        \end{center}
    \end{large}

      \vfill

    \begin{large}
      \begin{center}
        \textbf{PARA OPTAR AL TÍTULO DE: \tituloAOptarVal}\\[4mm]
      \end{center}
    \end{large}

    \begin{normalsize}
      \begin{center}
        \textbf{POSTULANTE: \autorPortadaVal}\\[0.3cm]
        \textbf{TUTOR: \nombreAsesorVal}\\[0.3cm]
      \end{center}
    \end{normalsize}
  
    \begin{normalsize}
      \begin{center}
        \textbf{\lugarPublicacionVal}\\
        \textbf{\fechaPublicacionVal}
      \end{center}
    \end{normalsize}
    
  \end{cambiamargen}

  \newpage

  % P\'agina que en la cara de detr\'as del folio de la portada.
  % Ponemos que el documento esta maquetado con TeXiS y la
  % aclaraci\'on de que debe imprimirse a doble cara.
  % \thispagestyle{empty}
  % \mbox{ }
  % \vfill%space*{4cm}
  % \begin{small} 
  %   \begin{center}
  %     \ifx\noTeXiSCreditsVal\undefined
  %       Documento maquetado con \texis\ v. \texisVer.
  %     \else
  %       \mbox{ }
  %     \fi
  %   \end{center}
  % \end{small}
  
  % \vspace*{2cm}
  
  % \begin{small} 
  %   \begin{center}
  %     \ifx\explicacionDobleCaraVal\undefined
  %     \mbox{ }
  %     \else
  %     \noindent Este documento est\'a preparado para ser imprimido a doble
  %     cara.
  %     \fi
  %   \end{center}
  % \end{small}

  % %%%
  % % Segunda portada
  % %%%

  % \newpage

  % \thispagestyle{empty}

  % \mbox{ }

  % \begin{Huge}
  %   \begin{center}
  %     \tituloPortadaVal
  %   \end{center}
  % \end{Huge}

  % \vfill

  % \begin{large}
  %   \begin{center}
  %     \textoPrimerSubtituloPortadaVal
  %     \\ \mbox{ } \\ \mbox{ } \\ 
  %     \textoSegundoSubtituloPortadaVal \\ [0.3em]
  %   \end{center}
  % \end{large}

  % \vfill

  % \begin{large}
  %   \begin{center}
  %     \textbf{\institucionVal}\\[0.2em]
  %         \mbox{ }  \\
  %     \textbf{\fechaPublicacionVal}
  %   \end{center}
  % \end{large}


  % \newpage
  % \thispagestyle{empty}
  % \mbox{ }

  % % Informaci\'on del ISBN y copyright
  % \vskip 13cm
  % \ifx\copyrightInfoVal\undefined
  % \mbox{ }
  % \else
  % Copyright \textcopyright\ \copyrightInfoVal
  % \fi
  % \vskip 3cm
  % \ifx\isbnVal\undefined
  % \mbox{ }
  % \else
  % ISBN \isbnVal
  % \fi

} % \newcommand{\makeCover}

% Variable local para emacs, para que encuentre el fichero
% maestro de compilaci\'on
%%%
%%% Local Variables:
%%% mode: latex
%%% TeX-master: "../Tesis.tex"
%%% End:
