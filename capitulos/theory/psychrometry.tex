\section{PSICROMETR\'IA}

Para el estudio de la psicrometr\'ia mediante la ecuaci\'on de gases ideales, seg\'un (), se hacen las siguientes suposici\'ones ():

\begin{itemize}
  \item La fase s\'olida o l\'iquida no contiene gases disueltos,
  \item la fase gaseosa puede ser tratado como una mezcla de gases ideales,
  \item cuando la mezcla y la fase condensada est\'an dadas a una presi\'on y temperatura, el equilibrio entre la fase condensada y la mezcla gaseosa, no est\'a influenciada por la presencia de otros componentes; significa que cuando se alcanza el equilibrio, la presi\'on parcial del vapor ser\'a igual a la presi\'on de saturaci\'on dada a la temperatura de la mezcla.
\end{itemize}

La relaci\'on psicrométrica viene dada por:
\begin{equation}
  \frac{h_{c}}{k_{c_s}^{'}} = r
\end{equation}

\eqVanDerWaals

\subsection{Relación de humedad o humedad absoluta}
\begin{align}
  \eqHumidityRatioFst{\HumidityRatio}
\end{align}

\begin{align}
  \label{eq:psy:HumidityRatio2form}
  \eqHumidityRatioSnd{\HumidityRatio}
\end{align}

Si se conoce la humedad absoluta $\HumidityRatio$, la composición de la cantidad de vapor puede calcularse a partir de la ecuación \eqref{eq:psy:HumidityRatio2form}, quedando la expresión de la siguiente manera:
\begin{equation}
  \eqCompVapor{}
\end{equation}

\subsection{Humedad Relativa}
Son diversas las definiciones que se ha establecido; la definición termodinámica de acuerdo a (), es la relación de la fugacidad del vapor en la mezcla respecto a la fugacidad de la mezcla gaseosa.

Se define como la relación de la presión parcial de vapor del agua respecto a la presión saturada a una temperatura dada.

\begin{equation}
  \eqRelativeHumidityFst{\RelativeHumidity}
\end{equation}

\subsection{Punto de Rocío}
\subsection{Temperatura de Bulbo Húmedo}
\subsection{Temperatura de Bulbo Seco}



\subsection{Retos y deventajas}
Para \parencite[p. 8]{Bell2017}, puede ocurrir la formación potencial de hidratos de gas seco; estos son, moléculas de gas atrapados en una red de moléculas de vapor; este fenómeno suele suceder a xxx presiónes.
\subsection{Otras ecuaciones}


