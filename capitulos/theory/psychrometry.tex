\section{PSICROMETRÍA}

  El grupo Europeo de Investigadores (GERG, por sus siglas en francés)  Algorithms Psychrometry, Bell, Harvey

Para el estudio de la psicrometría mediante la ecuación de gases ideales, según (), se hacen las siguientes suposiciones ():

\begin{itemize}
  \item La fase sólida o líquida no contiene gases disueltos,
  \item la fase gaseosa puede ser tratado como una mezcla de gases ideales,
  \item cuando la mezcla y la fase condensada están dadas a una presión y temperatura, el equilibrio entre la fase condensada y la mezcla gaseosa, no está influenciada por la presencia de otros componentes; significa que cuando se alcanza el equilibrio, la presión parcial del vapor ser\'a igual a la presión de saturaci\'on dada a la temperatura de la mezcla.
\end{itemize}

La relación psicrométrica viene dada por:
\begin{equation}
  \frac{h_{c}}{k_{c_s}^{'}} = r
\end{equation}

\eqVanDerWaals

\subsection{Humedad Absoluta o Relación de Humedad}
  La humedad absoluta se define como la relación de la masa del vapor \performVapor{} y la masa del gas \performGas{}:
  \begin{align}
    \eqHumidityRatioFst{\HumidityRatio}
  \end{align}

  Mediante el modelo de ecuación de estado del gas ideal (\eqIdealGasEosForMass{}):
\begin{align}
  \label{eq:psy:HumidityRatio2form}
  \eqHumidityRatioSnd{\HumidityRatio}
\end{align}

Si se conoce la humedad absoluta $\left(\HumidityRatio \right)$, la composición de la cantidad de vapor puede calcularse a partir de la ecuación \eqref{eq:psy:HumidityRatio2form}, quedando la expresión de la siguiente manera:
\begin{equation}
  \eqCompVapor{}
\end{equation}

\subsection{Mezcla Saturada Vapor-Gas}
  Cuando un gas seco entra en contacto con un líquido de un vapor, el líquido se evaporará hacia el gas hasta alcanzar el equilibrio, la presión parcial del vapor \performVapor{} en la mezcla vapor-gas alcanza su valor de saturación. la presión de vapor $\symPressure{\symVapor{}{}}{}$. El gas puede ser considerado como isoluble  en el líquido la naturaleza de la presión de vapor en la mezcla saturada es independiente de la naturaleza del gas y la presión total (excepto a altas presiones )  parencite{Treybal.Mass.Transfer.1981},   

\subsection{Mezcla Insaturada Vapor-Gas}

\subsection{Humedad Relativa}
    Representado mediante (\symRelativeHumidity), algunas de las {\color{blue!40!green} interpretaciones} que se ha establecido; para \parencite{Cusi.William} es un indicador del grado de saturación de un gas seco con un vapor,
    la definición termodinámica de acuerdo a (), es la relación de la fugacidad del vapor en la mezcla respecto a la fugacidad de la mezcla gaseosa.

    Se define como la relación de la presión parcial de vapor del agua respecto a la presión saturada a una temperatura dada.

    \begin{equation}
      \eqRelativeHumidityFst{\relativeHumidity}
    \end{equation}

    \begin{equation}
      \eqAbsoluteHumidityDevelop
    \end{equation}
    \namingEqAbsoluteHumidityDevelop

\subsection{Punto de Rocío}
\subsection{Temperatura de Bulbo Húmedo}
\subsection{Temperatura de Bulbo Seco}



\subsection{Retos y desventajas}
Para \parencite[p. 8]{Bell2017}, puede ocurrir la formación potencial de hidratos de gas seco; estos son, moléculas de gas atrapados en una red de moléculas de vapor; este fenómeno suele suceder a xxx presiónes.
\subsection{Otras ecuaciones}


