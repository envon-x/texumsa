% \section{El concepto de Estado en Termodinámica}

%%% OK
\section{ESTADOS DE LA MATERIA}
La termodinámica clásica se encarga de estudiar los estados de agregación: sólido, líquido y gaseoso. 
La fase gaseosa, donde las moléculas están bastante apartadas y sin orden aparente, adopta la forma del recipiente que los contiene \parencite[p. 111]{YunusA.Cengel2015}; 
en la fase líquida, las fuerzas intermoleculares son más débiles y también adopta la forma del recipiente  que los contiene. Para \parencite{ReyesChumacero2012}, las principales diferencias de los distintos estados de la materia están en el arreglo dinámico de sus partículas (posición, distancia y movimiento).

\subsection{ESTADO TERMODINÁMICO}
Un estado termodinámico queda definido por las \textit{propiedades termodinámicas} \parencite[26]{Smith.VanNess.Abbott2007}. Son  propiedades termodinámicas la presión, temperatura y el volumen; estas propiedades depende unos de otros.
\newline

Las propiedades pueden clasificarse en extensivas e intensivas. \newline
Las \textit{propiedades extensivas} son todas aquellas que depende de la cantidad de sustancia; es decir, su valor cambia con la cantidad de sustancia .
Las \textit{propiedades intensivas}, independiente de la cantidad de sustancia, este es invariante.

\subsection{FUNCIONES DE ESTADO}
Las propiedades termodinámicas 


\subsection{VARIABLES DE ESTADO}
Las variables de estado definen el estado de un sistema termodinámico. Para \parencite{castellan1998fisicoquimica}, el estado se define mediante las siguientes propiedades: masa, volumen, presión y temperatura. Mediante una { \it ecuación de estado}, puede determinarse la cuarta propiedad, el cual se conoce a partir del comportamiento del sistema.

% \subsection{Propiedades o Variables Termodinámicas}
% El estado termodinámico queda establecido mediante las propiedades termodinámicas como la termperatura, presión y densidad.
% Las cantidades intensivas que no depende a los sucesos que se haya sometido el sistema termodinámico; estas propiedades termodinámicas se denominan {\it funciones de estado} \parencite{Smith.VanNess.Abbott2007}


% antes cambio de estado: No se puede hablar de equilibrio sin antes definir el concepto de estado termodinámico
\subsection{EQUILIBRIO}
Según \parencite[p. 29]{Smith.VanNess.Abbott2007}, es la ausencia de cambio, lo cual denota una condición estática; bajo esta condición no puede ocurrir ningún cambio de estado e indica la ausencia de fuerzas impulsoras hacia un cambio de estado.

\subsection{EQUILIBRIO TERMODINÁMICO}
El equilibrio se como el estado de balance donde las fuerzas se cancelan mutuamente. Existen diferentes tipos de fuerzas, de los cuales se mencionan a continuación:
\begin{itemize}
  \item Equilibrio mecánico,
  \item equilibrio térmico,
  \item equilibrio químico,
  \item equilibrio de fases,
  \item equilibrio termodinámico.
\end{itemize}

Para \parencite{Reisel2021} y \parencite{YunusA.Cengel2015}, el equilibrio termodinámico sucede cuando un sistema alcanza el equilibrio mecánico, térmico, químico y el equilibrio de fases.
Un sistema en equilibrio termodinámico es incapaz de realizar un cambio espontaneo de su estado; cualquier cambio de estado recibe el nombre de proceso \parencite[p. 39]{YunusA.Cengel2015}.




\subsubsection{FASE}
    De acuerdo a \parencite[p. 30]{Smith.VanNess.Abbott2007}, una fase es una región homogénea de la materia; es decir, un gas, una mezcla de gases, un líquido, una solución de líquidos, un sólido cristalino son algunos ejemplos de fases. Para \parencite[p. 461]{ChangQmc2010}, una fase es una parte homogénea de un sistema, y aunque está en contacto con otras partes del sistema, está separada de esas partes por un límite bien definido.
    
    Pueden coexistir varias fases con la condición de que deben estar en equilibrio. A los líquidos y sólidos también se les conoce como fases condensadas o (estados condensados) \parencite[p. 430]{SilberberglQmcGeneral}. 

\subsection{CAMBIOS DE FASE}
    También conocido como transformaciones de fase, ocurre cuando se agrega o quita energía, generalmente en forma de calor. Los cambios de fase son cambios físicos que se distinguen porque  cambia el orden molecular, y en la fase gaseosa las moléculas presentan un mayor desorden \parencite[p. 489]{ChangQmc2010}. Los cambios entre los diferentes estados de agregación, según \parencite{ReyesChumacero2012}, se debe a las interacciones a nivel atómico y molecular de las sustancias y se manifiestan a nivel macroscópico como fuerza de {\it cohesión}, el cual sucede entre moléculas idénticas y las fuerzas de {\it adhesión} que se dan entre moléculas diferentes.

     
  % \figPhaseDiagram

De la figura \ref{fig:diag_phase}, la temperatura de equilibrio para la coexistencia de la fase Sólida-Líquida a una presión dada se llama, temperatura de fusión normal o temperatura de congelación normal; para la temperatura de equilibrio a una determinada presión entre la fase Líquido-Gas se llama {\it Temperatura de ebullición normal}. "Si el punto triple está a una presión superior a 1 , la sustancia no tiene una temperatura de congelación normal o una temperatura de ebullición normal, pero tiene una temperatura de sublimación normal a la que el sólido y el gas coexisten a una presión igual a 1 atmósfera", \parencite[pp. 28]{Mortimer2008} %47   

\subsection{Estructura Cristalina}
    La fase sólida se divide en dos categorías:
    % \begin{enumerate }
    %   \item sólidos cristalinos 
    %   % (las fuerzas que mantienen su estabilidad pueden ser iónicas,  covalentes, de van der Waals, enlaces de hidrógeno o una combinación de todas ellas.)
    %   \item sólidos amorfos (ej. vidrio)
    % \end{enumerate}

    \subsubsection*{Tipos de cristales}
    \begin{itemize}
      \item cristales iónicos (formado por cationes y aniones que suelen ser de distinto tamaño)
      \item cristales covalentes (se mantienen en una red tridimensional. Ej: diamante y grafito)
      \item cristales metálicos (los electrones están deslocalizado en todo el cristal)
    \end{itemize}


\subsection{Equilibrio Líquido-Vapor}
    

\subsubsection*{Tipos de sólidos amorfos}
    Se forma cuando los átomos y moléculas del fluido no tienen tiempo de alinearse por sí mismos y pueden quedar fijas en posiciones distintas a uno de cristal ordenado
    \begin{itemize}
      \item cristales iónicos (formado por cationes y aniones que suelen ser de distinto tamaño)
      \item cristales covalentes (se mantienen en una red tridimensional. Ej: diamante y grafito)
      \item cristales metálicos (los electrones están deslocalizado en todo el cristal)
    \end{itemize}


    Cuando las moléculas de un líquido tienen suficiente energía para escapar de la superficie, sucede un cambio de fase

\subsection{Presión de vapor}
    Cuando algunas moléculas de un líquido tienen energía suficiente para vender las fuerzas atractivas, y pasar al estado gaseoso, el proceso se llama evaporación o vaporización (Universidad de alcalá, cap 8)


\subsection{Mezclas Gas-Vapor}
Las mezclas debajo de la temperatura crítica


\subsection{Propiedades de los Fluidos}


\subsubsection{Regla de Fase}
    \begin{equation}
    \label{eq:thermo:phase_rule}
      \eqPhaseRule
    \end{equation}

    Donde,\newline 
    \freeDegree : es el número de grados de libertad o las variables que se tienen que especificar, \newline 
    \componentNumber : representa al número de componentes y \newline
    \phaseNumber : es el número de fases (sólido, líquido o gaseoso) en el sistema.
    % \includesvg[width=4.0in]{Imagenes/Vectorial/thermo/coord_P_V.svg}

  \figCoordinatePV

\subsubsection{Naturaleza de las mezclas Gas-Vapor}
el
\subsubsection{Naturaleza del Equilibrio}

\subsubsection{Equilibrio Dinámico}
\subsubsection{Sistema Termodinámico}

\subsubsection*{Modelo para el Equilibrio Líquido-Vapor}
\subsubsection*{Variables Macroscópicas Volumen, Presión, Temperatura}

\subsubsection{Temperatura}
En la actualidad se define en función a la constante de Boltzmann\footnote{Generado mediante \href{https://docs.scipy.org/doc/scipy/reference/constants.html}{cipy} y pythontex} {$k$}, \pcu{Boltzmann constant} en
\subsubsection{Presión}
En general, la presión se define como la fuerza normal ejercida sobre una unidad de area. Se habla de presión cuando el sistema consiste de fluidos (gas o líquido); pero cuando este es sólido, se usa el término \textit{esfuerzo normal} \textcite{YunusA.Cengel2015}

\begin{equation}
  \eqPressure
\end{equation}

\figMolecularColissions

de un sistema en fase gaseosa, es la presión que aporta uno de los componentes


\subsubsection{Presión Parcial}
\subsubsection{Presión Parcial}

\subsubsection{Presión Parcial}
como la fracción de uno de los componentes respecto a la suma de cada una de ellas, denominado presión total.

\eqGasCompositionFromPressure
Donde \symCompGV{i} es la composición molar, {\P{i}} es la presión del componente {${i}$}.

la expresión \eqref{eq:thermo:GasCompositionFromPressure}, también puede expresarse de la siguiente forma:

\eqGasCompositionFromTotalPressure
Donde

\subsubsection*{Ley de Raoult}
Se requiere hacer las siguientes suposiciones para las consideraciones para la ley de Raoult \parencite{Smith.VanNess.Abbott2007}
\begin{itemize}
  \item La fase de vapor es un \textbf{gas ideal}; significa que la aplicación se aplica solo a presiones bajas y moderadas.
  \item la fase líquida es una \textbf{solución ideal}; implica que la validez aproximada se da cuando las especies que constituyen el sistema, sean químicamente semejantes\footnote{Especies moleculares de similar tamaño y de la misma naturaleza química.}
\end{itemize}

En base a las suposiciones mencionadas se deduce la siguiente expresión matemática

\begin{equation}
  \eqRaoultLaw
\end{equation}

Donde {$\symCompGV{}_{i}$} es la composición molar de la fase vapor,  {\pressure{}{}} es la presión total de la mezcla, {$\symCompL{}_{i}$} representa a la composición molar de la fase líquida del sistema y

% obs
De la ecuación \eqref{eq:thermo:RaoultLaw}, la expresión del lado izquierdo se conoce como, presión parcial del componente ${i}$
% El gas ideal permite la comparación del comportamiento del gas real 


\begin{align}
  f_{i}^{\degree, \text{ fase vap}} & = \hat{\phi_{i}} \symCompGV{i} \pressure{}{} \\
  f_{i}^{\degree, \text{ fase liq}} & = \hat{\phi_{i}} \symCompL{i} p_{i}
\end{align}

\begin{flalign}
  f_{i}^{\degree, \text{ fase vap}} & = \hat{\phi_{i}} \symCompGV{i} \pressure{}{} \\
  f_{i}^{\degree, \text{ fase liq}} & = \hat{\phi_{i}} \symCompL{i} p_{i}
\end{flalign}

\begin{table}[htbp]
  \centering
  \caption{prueba}
  \rowcolors{1}{gray!20}{}
  \begin{tabular}{ccc}
    \hline
    \rowcolor{color_aquamarine} Variable & En la Ley de Raoult                              & En la Ley de Raoult modificada                               \\
    \hline
    $\hat{\phi_{i}}$                     & 1.00                                             & 1.00                                                         \\
    \symCompGV{i}                        & \symCompGV{i}                                    & \symCompGV{i}                                                \\
    $f_{i}^{\degree \text{, fase vap}}$  & \pressure{}{}                                             & \pressure{}{}                                                         \\
    ${\gamma}_{i}$                       & 1.00                                             & ${{\gamma}_{i}}$                                             \\
    $f_{i}^{\degree, \text{ fase liq}}$  & $p_{i}$                                          & $p_{i}$                                                      \\
    Ec. resultante                       & $\symCompGV{i} = \frac{\symCompL{i} {P_{i}}}{P}$ & $\symCompGV{i} = \frac{ \gamma_{i} \symCompL{i} {P_{i}}}{P}$ \\ \hline
  \end{tabular}
  % Tomado de: PHYSICAL AND CHEMICAL EQUILIBRIUM FOR CHEMICAL ENGINEERS

\end{table}


\subsubsection*{Ley De Henry}

%-------------------------------------------------------------------
Según \parencite{Engel2019}, una limitación de la ley de los gases ideales es que no predice bajo las condiciones apropiadas la licuefacción\footnote{cambio de estado de una sustancia cuando pasa del estado gaseoso al líquido} de gases.



\section{ECUACIONES DE ESTADO}
    Los estudios realizados por R. Boyle(1662), Charles () y Gay Lussac, a partir de sus estudios independientes, permitieron la generalización del comportamiento los gases mediante una relación matemática conocida como {\it ley de gas ideal}. 

\subsection{ECUACIÓN DE GAS IDEAL}
    El comportamiento ideal de un gas, se expresa mediante la siguiente ecuación de estado:

    \begin{equation}
      \eqIdealGas
    \end{equation}

    Donde {\zFactor} es el coeficiente de compresibilidad, {\pressure{}{}} es la presión absoluta, {\Vmolar} es el volumen molar, {\T{}} temperatura y {\univConstGasEos} es la constante universal de los gases.

\subsection{ECUACIÓN VIRIAL GENERAL}
    \label{cap2:subsec:tagEqnvirial}
    Para \parencite[p. 37]{Sengers1987equations}, la ecuación virial de estado de los gases, es una de las ecuaciones que intenta describir de manera exacta y precisa el comportamiento de los fluidos reales. 

    Una ventaja del uso de la ecuación virial es la existencia amplia de información experimental \Parencite[p. 38]{Sengers1987equations}, 

    La ecuación virial puede ser apreciada como una serie de Maclaurin\footnotetext{También llamado serie de Taylor}, donde ${\pressure{}{}}/{\univConstGasEos \T{}}$; en vista de su naturaleza polinomial, puede escribirse como serie de potencias de $1/\Vmolar$, el cual se muestra a continuación:

    \begin{equation}
      \eqVirialForVDevelop{}
    \end{equation}
    \eqVirialForVDevelopNaming
    
    El 2do y el 3er coeficiente virial ha sido estudiado por más de 100 años, todos ellos muestran que no existe correlación \parencite{thirdVirialTing}


    % Se considera las siguientes suposiciones:

  
\subsubsection*{Dependencia de los coeficientes viriales de la temperatura}
    \begin{equation}
      \bCoeffDependentOfT{}
    \end{equation}
    
    \begin{equation}
      \cCoeffDependentOfT{}
    \end{equation}

\subsubsection*{Dependencia de los coeficientes viriales de la composición}
    Para un gas puro, de acuerdo a la ecuación \eqref{eq:thermo:phase_rule} se necesita, dos grados de libertad (\freeDegree),  

    Donde, $x = 1$ 

\subsubsection{Métodos para la estimación de los coeficientes viriales}
    Diversos autores señalan que los coeficientes viriales pueden derivarse de \parencite[p. 4.13]{Poling2001} de la teoría molecular
    Para \parencite{Trusler10.1039/9781782627043-00152}, los métodos para la medición de los coeficientes viriales pueden clasificarse en:

    \begin{enumerate}[label=(\alph*)]
      \item Métodos directos
        \begin{itemize}
          \item Aparato de Burnet
          \item Densímetro de doble plomada acoplados magnéticamente
        \end{itemize}
      \item  Métodos indirectos
        \begin{itemize}
          \item Flujo calorimétrico
          \item Velocidad del sonido
        \end{itemize}
    \end{enumerate}

    para \parencite{Dymond2002}, los métodos para la determinación de los coeficientes viriales pueden clasificarse de la siguiente manera:

    \begin{itemize}
      \item Medidas de {\pressure{}{}-\Vmolar-\T{}},
      \item Medidas de la velocidad del sonido
      \item Medidas de Joule-Thomson
      \item Medidas de índice de refractividad y permitividad relativa
      \item Medición de la entalpía de vaporización y la presión de vapor. 
    \end{itemize}

\subsubsection{Estimación del Segundo Coeficiente Virial}
  
    El segundo coeficiente virial representa a la interacción entre pares de moléculas

%%%%
Son diversos los trabajos que se han realizado para determinar el segundo coeficiente virial. \autocite[postnote]{woolley1969second} trabajó para la formulación analítica del segundo coeficiente virial para una función esféricamente simétrica al cual denominó, \textit{potencial de par realista}. \parencite{pitzer1990second} también trabajó para estimar el segundo coeficientes virial para compuestos no polares y de baja polaridad. Para compuestos polares se han realizado diversos estudios para estimar el segundo coeficiente virial, entre los cuales destacan los estudios de \parencite{maris1985interaction}, \parencite{tarakad1977improved}, y entre otros estudios se han realizado . \parencite{vetere2007simple} propuso una modificacion a método de Pitzer para predecir el segundo coeficiente virial para compuestos puros.

\parencite{klotz1985improved} presenta una mejora para el cálculo del 2do coeficiente virial



Para la temperatura $\T{}$ y composición molar $\symCompGV{}$, ambos constantes; de la ecuación \eqref{eq:virialB}, cuando $1 / \Vmolar\ \to 0$,  el segundo coeficiente virial $\bCoeffVirial{}$, puede escribirse como:

\begin{equation}
  \bCoeffVirial{} = \lim_{\frac{1}{\Vmolar}\ \to {0}} \Vmolar \left(\frac{\pressure{}{} \Vmolar} {\univConstGasEos \T{}} - 1 \right)
\end{equation}

Para \parencite[p. 473]{beattie1942second}, la expresión $\Vmolar \left(\frac{\pressure{}{} \Vmolar} {\univConstGasEos \T{}} - 1 \right)$, puede ser calculado a partir de las mediciones de la compresibilidad isotérmica de un gas puro; en caso de una mezcla, debe ser de composición molar constante; la compresibilidad isotérmica debe establecerse frente a $1 / \Vmolar$. El valor de la constante queda establecido en la intersección de la función con el eje $1/ \Vmolar$.

Para una mezcla binaria, $\bCoeffVirial{}$ está relacionado con los coeficientes viriales de los componentes puros y sus composiciones, de la siguiente manera:

\begin{equation}
  \eqSecondVirialCoeffSum{}
\end{equation}

Para un sistema de dos componentes ($\symNumOfCmpnts = 2$),
\begin{equation}
  \eqSecondVirialCoeffDevelop{}
\end{equation}
\eqSecondVirialCoeffNaming{}

Conociendo los coeficientes viriales para los componentes puros, hace falta establecer las relaciones matemáticas para {\bCoeffVirial{\nameG \nameV}}.



\subsubsection{Estimación del Tercer Coeficiente Virial}
Para \parencite{Dymond2002}, los valores son preferentemente determinados mediante la información de compresibilidad de gases 

\begin{equation}
  \cCoeffVirial{} = \lim_{\frac{1}{\Vmolar}\ \to {0}} \left[ \Vmolar \left(\frac{\pressure{}{} \Vmolar} {\univConstGasEos \T{}} - 1 \right) - \bCoeffVirial{} \right] \Vmolar
\end{equation}


\begin{equation}
  \eqThirdVirialCoeffSum{}
\end{equation}

Desarrollando para un sistema de dos componentes:

\begin{equation}
  \thirdVirialCoeffDevelop{}
\end{equation}

\eqThirdVirialCoeffNaming{}

\figVirialBCBehavior

\begin{align}
  \eqVirialFactorCorrection{}
\end{align}

\factorCorrectionNaming{}

\subsection{Convergencia de la serie virial}

\subsection{Teoría del Sonido}
\subsubsection*{Velocidad del Sonido en los Gases}

\parencite[p. 186]{Wilhelm2010}, define la velocidad del sonido como:

\begin{align}
  \label{cap:subsec:velocidadSonido}
  \SoundSpeed{}
\end{align}

Donde $\pressure{}{}$  es la presión, $\Density$ es la densidad, $\Entropy$ es la entropía

\subsection{Correlación, predicción y estimación de los coeficientes viriales}
\label{cap2:subsec:tagEqnPredict}




Según \parencite[ p. 12]{Dymond2002}, el método basado en la teoría del sonido, es el método que mejor se acerca a los valores reales; de esta manera, el segundo coeficiente virial queda expresado de la siguiente forma:
\begin{equation}
  \eqSndVirialCoeff{1}{2}{\T{}}
\end{equation}


Donde $f_{12}$ = $ \ce{exp}{\{-U(R, \omega_1 , \omega_2)/kt\}} - 1 $. Para una molécula lineal:

Desarrollo de las ecuaciones para el cálculo de los coeficientes viriales fluidos polares y no polares





\subsection{VOLUMEN MOLAR PARA UNA MEZCLA}
    Para una mezcla el volumen molar
  \begin{equation}
    \eqVirialForVDevelopForMixture
  \end{equation}
    
    \eqVirialForVDevelopMixtureNaming