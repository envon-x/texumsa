\section{OBJETIVOS}
\label{cap1:sec:objetivos}

\subsection{OBJETIVO GENERAL}
\begin{itemize}
    \item Desarrollar una aplicación para la plataforma Android que permita visualizar cartas psicrométricas para múltiples mezclas vapor del líquido y gas seco, relacionada con aplicaciones ambientales. %en un solo punto. %para más de una mezcla gas-vapor.
\end{itemize}
\subsection{OBJETIVOS ESPECÍFICOS}
    \begin{itemize}
        % \item Desarrollar una aplicación para múltiples mezclas vapor-gas la información que permita construir las cartas psicrometétricas hospedados en la nube en base a una combinación de variables.
        % \item La variable principal será presión atmosférica.
        % \item Proveer a los técnicos en el area de ingeniería de una app facil de leer las variables psicrométricas
        % \item Publicar el código fuente bajo la licencia Apache v2.0
        % \item Atenuar los colores de las lineas para una mejor lectura
         
        % \item Desarrollar una aplicación de forma que permita la visualización del diagrama psicrométrico para mezclas diferentes de vapor-gas.
        \item Desarrollar la aplicación en el lenguage de programación \ldp, de manera que permita la extensión a mezclas diferentes vapor del líquido - gas seco.
        \item Implementar para la mezcla vapor de agua-aire seco, de manera predeterminada.
        % \item Implementar para la mezcla de vapor de $\ch{CO2}$ y aire seco.
        % \item Implementar el diagrama psicrométrico para la mezcla, Benceno - Nitrogenó.
        \item Desarrollar algoritmos para la determinación de las variables psicrométricas dependientes: Humedad absoluta, Humedad relativa, temperatura de bulbo húmedo, temperatura de rocío, volumen específico. 
        
        % \item Mostrar los resultados obtenidos de las variables psicrométricas en un cuadro de visualización para su lectura inmediata.
        % \item Implementar para el sistema de unidad internacional, imperial y personalizadas
        


        %adicionar despues XXX
        % \item Implementar la aplicación para el \acrfull{si} de unidades y otras unidades de interés en las ciencias ambientales. % y personalizadas. % selección dinámica de unidades
        

        % \item Implementar para el \acrfull{si} de unidades y unidades convencionales relacionadas al area de meteorología y climatología. % y personalizadas. % selección dinámica de unidades
        \item Comparar los resultados obtenidos a partir de la aplicación desarrollada con el simulador \href{https://dwsim.fossee.in}{DWSIM}.
        \item Justificar el uso de la ecuacion de gases ideales respecto a la ecuación virial de gases para las mezclas de vapor de agua $-$ aire seco. %, a una presión definida.

        \item Presentar a Google para su comprobación y aprobación de la originalidad del  documento terminado.
        \item Mostrar el producto final en la tienda oficial de aplicaciones de Google, PlayStore.
        \item Desarrollar dos casos de aplicación de la herramienta en el area de Ingeniería Ambiental 
        % \item Implementar 
        % \item Escribir código por lo menos para 3 mezclas binarias gas-vapor.
        % \item Mostrar dinámicamente las coordenadas
        % \item Implementar el diagrama psicrométrico para una mezcla extra vapor-gas.
        % \item Comparar los resultados obtenidos con la app, la regla psicrométrica que utilizan las entidades del estado, tales como \acrfull{senamhi} y una algoritmo externo.
        % \item Comparar la diferencia en el aprendizaje de tópicos relacionados a psicrometría con la aplicación y sin ella. %entre los resultados 
        % \item Identificar las razones del uso de reglas  manuales psicrométricas dentro del \acrfull{senamhi}, en base a ello, realizar sugerencias a dicha institución.
        
        % \item Sugererir a la \acrfull{rae}, la corrección del significado dado a la palabra, \textit{climatizar}.
        % \item Comparar los resultados obtenidos con la app, la regla psicrométrica que utilizan las entidades del estado, tales como \acrfull{senamhi} y una algoritmo implementado externo
        % \item Disponer de serie de diagramas para la mezcla de gas-vapor en un solo punto y acceder a ellas desde cualquier punto (disponible para cualquiera que desee consultar),
        % \item buscador, numero de búsquedas, tiempo que les a tomado, tipo de pc, provedor de internet, calidad del producto encontrado (tiempo de apendizaje con la app y sin app, formato de la imagen pdf, jpeg. fuente de obtencion, elaboracion, link, sistema operativo)
   
    \end{itemize}