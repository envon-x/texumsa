\documentclass[letterpaper, 12pt, oneside]{book}
    \usepackage{resources/texumsa/styles/tesis}
    \usepackage{resources/physics/styles/impl-equations}

    % \usepackage{resources/algorihtms/styles/eos-algorithm}
    \usepackage{figuras/viriales}
    \usepackage{figuras/cinetic-models}
    
    \usepackage{pdfpages}
    \usepackage{setspace} %espacio entre lineas: http://minisconlatex.blogspot.com/2015/02/como-editar-el-interlineado-en-latex.html  
    % \renewcommand{\baselinestretch}{1.5}
    % \doublespacing  % \onehalfspace  % \singlespace  % \spacing{1.5}

    \usepackage[english, spanish,es-noindentfirst,]{babel}
    \usepackage[T1]{fontenc}
    \usepackage[utf8]{inputenc}
    \usepackage{csquotes}
    \usepackage{threeparttable}
    
    \usepackage{custom-bibliography}

    % Para tabla y figuras formato apa
    \usepackage{multirow,booktabs,setspace,caption}
    \usepackage{tikz}
    \usepackage{wrapfig} % Para alinear figuras pequeñas dentro del texto
    \usepackage{media9} % Para adicionar imagenes GIF
    \usepackage{qrcode} %QR images

    \usepackage{graphicx}

    % physcla units and values
    % \usepackage{mandi} % error Luatex
    % \usepackage{pythontex}

    \usepackage[normalem]{ulem} % para tachar lineas cancela
    \usepackage{xcolor}
    \newcommand{\soutthick}[1]{%
        \renewcommand{\ULthickness}{1.4pt}%
        \sout{#1}%
        \renewcommand{\ULthickness}{0.4pt}% Resetting to ulem default
    }

    % para la corrección al estilo apa de las etiquetas apa7
    \DeclareCaptionLabelSeparator*{spaced}{\\[2ex]}
    \captionsetup[table]{textfont=it, format=plain, justification=justified, singlelinecheck=false,labelsep=spaced, skip=0pt}
    \captionsetup[figure]{labelfont=bf, format=plain, justification=justified, singlelinecheck=false, font=small}
    % \captionsetup[figure]{labelsep=period,labelfont=it,justification=justified, singlelinecheck=false,font=doublespacing}

    \usepackage{graphicx} % el argumento dvips, para incluir graficos eps

    \urlstyle{same}

    \usepackage{glossaries}
    % \usepackage[xindy, nonumberlist, toc, nopostdot, style=altlist, nogroupskip]{glossaries} % para gloasarios y acronimos
    \usepackage[record,acronym,postdot]{glossaries-extra}%para abreviaturas % https://tex.stackexchange.com/questions/473744/manage-multiple-glossaries-in-one-bib-file

    \graphicspath{{images/}}

    \usepackage{blindtext}
    % \usepackage{subfiles} % Best loaded last in the preamble

    % \usepackage{turnthepage}
    %\usepackage{pdflscape}
    \usepackage[paper=portrait,pagesize]{typearea} % Only rotating one PDF page for table
    \usepackage{colortbl} % color de las celdas para tablas

    \usepackage{threeparttable} % Only rotating one page for table
    \usepackage{booktabs}
    \usepackage{appendix}
    \renewcommand{\appendixpagename}{Apéndices}
    \renewcommand{\appendixtocname}{Apéndices} % Nombre que aparece para los apendices en el índice

    % https://ondiz.github.io/cursoLatex/Contenido/10.DocumentoCientifico.html
    \usepackage[draft]{listofsymbols} % cambiar 'draft' por 'final' en el definitivo

    %Cambiar Cuadros por Tablas y lista de...
    \renewcommand{\listtablename}{Índice de cuadros}

    \usepackage{enumitem}                                               % Para enumerar listas con letras, números romanos


    % Para numerar las figuras apropiadamente - no corrige el error de referencias el numero de esas figuras
    \usepackage{chngcntr}
    \counterwithin{figure}{section}

  

    % Para cambiar el tamaño de las fuentes del titulo de las secciones
    % \rmfamily selects a roman (i.e., serifed) font family
    % \sffamily selects a sans serif font family
    % \ttfamily selects a monospaced (“typewriter”) font family
    % \usepackage{sectsty}
    % \allsectionsfont{\rmfamily} 
    % \sectionfont{\fontsize{12}{15}\selectfont}
    
    \loadglsentries{db_glossary}
    \loadglsentries{db_acronyms}
    \GlsXtrLoadResources[src={db_abreviations.bib}]
     
    %ver documentacion glossaries=extra-manual.pdf page 193
    % \GlsXtrLoadResources[src={db_abreviations.bib, biblio}] %ver documentacion glossaries=extra-manual.pdf page 193
    %%%%%%%%%%%%%%%%%%%%%%%%%%%%%%%%%%%%%%%%%%%%%%%%%%%%%%%%%%%%%%%%%%%%%%%%%%%%%%%%%%%%%%%%%%%%%%%%%%%%%%%%%%%%%%%%%%%%%%%%%%%%%%%%%%%%%%%%%%%%%%%%%%%%%%%%%%%%%%%%%%%%%%%%



\begin{document}
  % \layout % Esto genera un diagrama con los nombres y tamaños de las partes de la página

  \frontmatter
  
%%%%%%%%%%%%%%%%%%%%%%%%%%%%%%%%%%%%%%%%%%%%%%%%%%%%%%%%%%%%%%%%%%%%%%
% Institución/departamento asociado al documento.
% \institucion{Nombre}
% Puede tener varias líneas. Se utiliza en las dos portadas.
% Si no se indica aparecerá vacío.
%%%%%%%%%%%%%%%%%%%%%%%%%%%%%%%%%%%%%%%%%%%%%%%%%%%%%%%%%%%%%%%%%%%%%%
\institucion{%
  UNIVERSIDAD MAYOR DE SAN ANDRÉS
}

\facultad{
  FACULTAD DE INGENIERÍA
}

\carrera{
  INGENIERÍA QUÍMICA AMBIENTAL ALIMENTOS Y PETROQUÍMICA
}

%%%%%%%%%%%%%%%%%%%%%%%%%%%%%%%%%%%%%%%%%%%%%%%%%%%%%%%%%%%%%%%%%%%%%%
% Imagen de la portada (y escala)
% \imagenPortada{Fichero}
% \escalaImagenPortada{Numero}
% Si no se especifica, se utiliza la imagen TODO.pdf
%%%%%%%%%%%%%%%%%%%%%%%%%%%%%%%%%%%%%%%%%%%%%%%%%%%%%%%%%%%%%%%%%%%%%%
\imagenPortada{resources/images//vectorial/umsa-logo.pdf}
% \escalaImagenPortada{.13}



%%%%%%%%%%%%%%%%%%%%%%%%%%%%%%%%%%%%%%%%%%%%%%%%%%%%%%%%%%%%%%%%%%%%%%
% Tipo de documento.
% \tipoDocumento{Tipo}
% Para el texto justo debajo del escudo.
% Si no se indica, se utiliza "TESIS DOCTORAL".
%%%%%%%%%%%%%%%%%%%%%%%%%%%%%%%%%%%%%%%%%%%%%%%%%%%%%%%%%%%%%%%%%%%%%%
\tipoDocumento{
  \textbf{
  \fontsize{11}{0}\selectfont {
      PROYECTO DE GRADO    
    }
  }
}



\tituloInvestigacion{%
  \textbf{
  \fontsize{16}{0}\selectfont {
      IMPLEMENTACIÓN PARA LA PLATAFORMA ANDROID DEL DIAGRAMA PSICROMÉTRICO PARA MEZCLAS BINARIAS DE VAPOR Y GAS SECO
    }
  }
  \\[1mm]     
}

\tituloAcademico{
    Ingeniero Ambiental
}
%%%%%%%%%%%%%%%%%%%%%%%%%%%%%%%%%%%%%%%%%%%%%%%%%%%%%%%%%%%%%%%%%%%%%%
% Autor del documento:
% \autorPortada{Nombre}
% Se utiliza en la portada y en el valor por defecto del
% primer subtítulo de la segunda portada.
%%%%%%%%%%%%%%%%%%%%%%%%%%%%%%%%%%%%%%%%%%%%%%%%%%%%%%%%%%%%%%%%%%%%%%
\autorPortada{HECTOR RAUL BONIFACIO CONDORI}
\asesorInvestigacion{MSc. Ing. OMAR SALINAS PRUDENCIO}
\lugarPublicacion{LA PAZ-BOLIVIA}

%%%%%%%%%%%%%%%%%%%%%%%%%%%%%%%%%%%%%%%%%%%%%%%%%%%%%%%%%%%%%%%%%%%%%%
% Fecha de publicación:
% \fechaPublicacion{Fecha}
% Puede ser vacío. Aparece en la última línea de ambas portadas
%%%%%%%%%%%%%%%%%%%%%%%%%%%%%%%%%%%%%%%%%%%%%%%%%%%%%%%%%%%%%%%%%%%%%%
\fechaPublicacion{Agosto, 2024}



%%%%%%%%%%%%%%%%%%%%%%%%%%%%%%%%%%%%%%%%%%%%%%%%%%%%%%%%%%%%%%%%%%%%%%
% Director del trabajo.
% \directorPortada{Nombre}
% Se utiliza para el valor por defecto del segundo subtítulo, donde
% se indica quién es el director del trabajo.
% Si se fuerza un subtítulo distinto, no hace falta definirlo.
%%%%%%%%%%%%%%%%%%%%%%%%%%%%%%%%%%%%%%%%%%%%%%%%%%%%%%%%%%%%%%%%%%%%%%
%\directorPortada{Walterio Malatesta}

%%%%%%%%%%%%%%%%%%%%%%%%%%%%%%%%%%%%%%%%%%%%%%%%%%%%%%%%%%%%%%%%%%%%%%
% Texto del primer subtítulo de la segunda portada.
% \textoPrimerSubtituloPortada{Texto}
% Para configurar el primer "texto libre" de la segunda portada.
% Si no se especifica se indica "Memoria que presenta para optar al
% título de Doctor en Informática" seguido del \autorPortada.
%%%%%%%%%%%%%%%%%%%%%%%%%%%%%%%%%%%%%%%%%%%%%%%%%%%%%%%%%%%%%%%%%%%%%%
% \textoPrimerSubtituloPortada{%
% \textit{Informe técnico del departamento}  \\ [0.3em]
% \textbf{Ingeniería del Software e Inteligencia Artificial} \\ [0.3em]
% \textbf{IT/2009/3}
% }

%%%%%%%%%%%%%%%%%%%%%%%%%%%%%%%%%%%%%%%%%%%%%%%%%%%%%%%%%%%%%%%%%%%%%%
% Texto del segundo subtítulo de la segunda portada.
% \textoSegundoSubtituloPortada{Texto}
% Para configurar el segundo "texto libre" de la segunda portada.
% Si no se especifica se indica "Dirigida por el Doctor" seguido
% del \directorPortada.
%%%%%%%%%%%%%%%%%%%%%%%%%%%%%%%%%%%%%%%%%%%%%%%%%%%%%%%%%%%%%%%%%%%%%%
% \textoSegundoSubtituloPortada{%
% \textit{Versión \texisVer}
% }

%%%%%%%%%%%%%%%%%%%%%%%%%%%%%%%%%%%%%%%%%%%%%%%%%%%%%%%%%%%%%%%%%%%%%%
% \explicacionDobleCara
% Si se utiliza, se aclara que el documento está preparado para la
% impresión a doble cara.
%%%%%%%%%%%%%%%%%%%%%%%%%%%%%%%%%%%%%%%%%%%%%%%%%%%%%%%%%%%%%%%%%%%%%%
% \explicacionDobleCara

%%%%%%%%%%%%%%%%%%%%%%%%%%%%%%%%%%%%%%%%%%%%%%%%%%%%%%%%%%%%%%%%%%%%%%
% \isbn
% Si se utiliza, aparecerá el ISBN detrás de la segunda portada.
%%%%%%%%%%%%%%%%%%%%%%%%%%%%%%%%%%%%%%%%%%%%%%%%%%%%%%%%%%%%%%%%%%%%%%
% \isbn{978-84-692-7109-4}


%%%%%%%%%%%%%%%%%%%%%%%%%%%%%%%%%%%%%%%%%%%%%%%%%%%%%%%%%%%%%%%%%%%%%%
% \copyrightInfo
% Si se utiliza, aparecerá información de los derechos de copyright
% detrás de la segunda portada.
%%%%%%%%%%%%%%%%%%%%%%%%%%%%%%%%%%%%%%%%%%%%%%%%%%%%%%%%%%%%%%%%%%%%%%
% \copyrightInfo{\autor}


%%
%% Creamos las portadas
%%
\makeCover

% Variable local para emacs, para que encuentre el fichero
% maestro de compilación
%%%
%%% Local Variables:
%%% mode: latex
%%% TeX-master: "../Tesis.tex"
%%% End:


    % Índices
  \pagenumbering{roman}
  
  \renewcommand\contentsname{\large Índice General}           % Contenido
  \thispagestyle{plain}                                       % para que la cabecera esté en blanco
  \tableofcontents
  \setcounter{tocdepth}{2}
  
  %---------------------------------------------------------------------
%
%                      dedicatoria.tex
%
%---------------------------------------------------------------------
%
% dedicatoria.tex
% Copyright 2009 Marco Antonio Gomez-Martin, Pedro Pablo Gomez-Martin
%
% This file belongs to the TeXiS manual, a LaTeX template for writting
% Thesis and other documents. The complete last TeXiS package can
% be obtained from http://gaia.fdi.ucm.es/projects/texis/
%
% Although the TeXiS template itself is distributed under the 
% conditions of the LaTeX Project Public License
% (http://www.latex-project.org/lppl.txt), the manual content
% uses the CC-BY-SA license that stays that you are free:
%
%    - to share & to copy, distribute and transmit the work
%    - to remix and to adapt the work
%
% under the following conditions:
%
%    - Attribution: you must attribute the work in the manner
%      specified by the author or licensor (but not in any way that
%      suggests that they endorse you or your use of the work).
%    - Share Alike: if you alter, transform, or build upon this
%      work, you may distribute the resulting work only under the
%      same, similar or a compatible license.
%
% The complete license is available in
% http://creativecommons.org/licenses/by-sa/3.0/legalcode
%
%---------------------------------------------------------------------
%
% Contiene la página de dedicatorias.
%
%---------------------------------------------------------------------

\dedicatoriaUno{%
\emph{
Con enorme amor y agradecimiento, dedico este trabajo 
a todas las personas que contribuyeron en la formación
académica, sobre todo a mis padres Gerónimo y Saturnina\\
y a mis hermanos Nelson, Jhonsson y Yesica.%
}%
}

% \dedicatoriaDos{%
% \emph{%
% A toda la comunidad del software libre que hizo posible este trabajo, 
% en especial a Donald Knuth.%
% }%
% }

\makeDedicatorias

% Variable local para emacs, para que encuentre el fichero
% maestro de compilación
%%%
%%% Local Variables:
%%% mode: latex
%%% TeX-master: "../Tesis.tex"
%%% End:

  
\chapter{Agradecimientos}

\cabeceraEspecial{Agradecimientos}

\begin{FraseCelebre}
\begin{Frase}
A todos los que la presente vieren y entendieren.
\end{Frase}
\begin{Fuente}
Inicio de las Leyes Orgánicas. Juan Carlos I
\end{Fuente}
\end{FraseCelebre}

Groucho Marx decía que encontraba a la televisión muy educativa porque
cada vez que alguien la encendía, él se iba a otra habitación a leer
un libro. Utilizando un esquema similar, nosotros queremos agradecer
al Word de Microsoft el habernos forzado a utilizar \LaTeX. Cualquiera
que haya intentado escribir un documento de más de 150 páginas con
esta aplicación entenderá a qué nos referimos. Y lo decimos porque
nuestra andadura con \LaTeX\ comenzó, precisamente, después de
escribir un documento de algo más de 200 páginas. Una vez terminado
decidimos que nunca más pasaríamos por ahí. Y entonces caímos en
\LaTeX.

Es muy posible que hubíeramos llegado al mismo sitio de todas formas,
ya que en el mundo académico a la hora de escribir artículos y
contribuciones a congresos lo más extendido es \LaTeX. Sin embargo,
también es cierto que cuando intentas escribir un documento grande
en \LaTeX\ por tu cuenta y riesgo sin un enlace del tipo ``\emph{Author
  instructions}'', se hace cuesta arriba, pues uno no sabe por donde
empezar.

Y ahí es donde debemos agradecer tanto a Pablo Gervás como a Miguel
Palomino su ayuda. El primero nos ofreció el código fuente de una
programación docente que había hecho unos años atrás y que nos sirvió
de inspiración (por ejemplo, el fichero \texttt{guionado.tex} de
\texis\ tiene una estructura casi exacta a la suya e incluso puede
que el nombre sea el mismo). El segundo nos dejó husmear en el código
fuente de su propia tesis donde, además de otras cosas más
interesantes pero menos curiosas, descubrimos que aún hay gente que
escribe los acentos españoles con el \verb+\'{\i}+.

No podemos tampoco olvidar a los numerosos autores de los libros y
tutoriales de \LaTeX\ que no sólo permiten descargar esos manuales sin
coste adicional, sino que también dejan disponible el código fuente.
Estamos pensando en Tobias Oetiker, Hubert Partl, Irene Hyna y
Elisabeth Schlegl, autores del famoso ``The Not So Short Introduction
to \LaTeXe'' y en Tomás Bautista, autor de la traducción al español. De
ellos es, entre otras muchas cosas, el entorno \texttt{example}
utilizado en algunos momentos en este manual.

También estamos en deuda con Joaquín Ataz López, autor del libro
``Creación de ficheros \LaTeX\ con {GNU} Emacs''. Gracias a él dejamos
de lado a WinEdt y a Kile, los editores que por entonces utilizábamos
en entornos Windows y Linux respectivamente, y nos pasamos a emacs. El
tiempo de escritura que nos ahorramos por no mover las manos del
teclado para desplazar el cursor o por no tener que escribir
\verb+\emph+ una y otra vez se lo debemos a él; nuestro ocio y vida
social se lo agradecen.

Por último, gracias a toda esa gente creadora de manuales, tutoriales,
documentación de paquetes o respuestas en foros que hemos utilizado y
seguiremos utilizando en nuestro quehacer como usuarios de
\LaTeX. Sabéis un montón.

Y para terminar, a Donal Knuth, Leslie Lamport y todos los que hacen y
han hecho posible que hoy puedas estar leyendo estas líneas.

\endinput
% Variable local para emacs, para  que encuentre el fichero maestro de
% compilación y funcionen mejor algunas teclas rápidas de AucTeX
%%%
%%% Local Variables:
%%% mode: latex
%%% TeX-master: "../Tesis.tex"
%%% End:
  \chapter{Resumen}
\cabeceraEspecial{Resumen}



\texis\ es un conjunto de ficheros \LaTeX\ que pueden servir para
escribir tesis doctorales, trabajos de fin de master, de fin de
carrera y otros documentos del mismo estilo. El documento que tienes
en tus manos es un manual que explica las distintas características de
la plantilla. En los distintos capítulos iremos explicando los
ficheros existentes en \texis\ así como su función. También se
explican algunas de las características, como por ejemplo ciertos
comandos que facilitan la escritura de los documentos.

Aunque el código \LaTeX\ utilizado en \texis\ está muy comentado
para su uso fácil, creemos que las explicaciones que aquí se
proporcionan pueden ser útiles.


  \chapter{Abstract}
\cabeceraEspecial{Abstract}

Psychrometry is a large environmental science...



  \ifx\generatoc\undefined
  \else
    %---------------------------------------------------------------------
%
%                          TeXiS_toc.tex
%
%---------------------------------------------------------------------
%
% TeXiS_toc.tex
% Copyright 2009 Marco Antonio Gomez-Martin, Pedro Pablo Gomez-Martin
%
% This file belongs to TeXiS, a LaTeX template for writting
% Thesis and other documents. The complete last TeXiS package can
% be obtained from http://gaia.fdi.ucm.es/projects/texis/
%
% This work may be distributed and/or modified under the
% conditions of the LaTeX Project Public License, either version 1.3
% of this license or (at your option) any later version.
% The latest version of this license is in
%   http://www.latex-project.org/lppl.txt
% and version 1.3 or later is part of all distributions of LaTeX
% version 2005/12/01 or later.
%
% This work has the LPPL maintenance status `maintained'.
% 
% The Current Maintainers of this work are Marco Antonio Gomez-Martin
% and Pedro Pablo Gomez-Martin
%
%---------------------------------------------------------------------
%
% Contiene  los  comandos  para  generar los  \'Indices  del  documento,
% entendiendo por \'Indices las tablas de contenidos.
%
% Genera  el  \'Indice normal  ("tabla  de  contenidos"),  el \'Indice  de
% figuras y el de tablas. También  crea "marcadores" en el caso de que
% se esté compilando con pdflatex para que aparezcan en el PDF.
%
%---------------------------------------------------------------------


% Primero un poquito de configuración...


% Pedimos que inserte todos los ep\'Igrafes hasta el nivel \subsection en
% la tabla de contenidos.
\setcounter{tocdepth}{2} 

% Le  pedimos  que nos  numere  todos  los  ep\'Igrafes hasta  el  nivel
% \subsubsection en el cuerpo del documento.
\setcounter{secnumdepth}{3} 


% Creamos los diferentes \'Indices.

% Lo primero un  poco de trabajo en los marcadores  del PDF. No quiero
% que  salga una  entrada  por cada  \'Indice  a nivel  0...  si no  que
% aparezca un marcador "\'Indices", que  tenga dentro los otros tipos de
% \'Indices.  Total, que creamos el marcador "\'Indices".
% Antes de  la creación  de los \'Indices,  se añaden los  marcadores de
% nivel 1.

\ifpdf
   \pdfbookmark{\'Indices}{\'indices}
\fi

% Tabla de contenidos.
%
% La  inclusión  de '\tableofcontents'  significa  que  en la  primera
% pasada  de  LaTeX  se  crea   un  fichero  con  extensión  .toc  con
% información sobre la tabla de contenidos (es conceptualmente similar
% al  .bbl de  BibTeX, creo).  En la  segunda ejecución  de  LaTeX ese
% documento se utiliza para  generar la verdadera página de contenidos
% usando la  información sobre los  cap\'Itulos y demás guardadas  en el
% .toc
\ifpdf
   \pdfbookmark[1]{Tabla de contenidos}{tabla de contenidos}
\fi

\cabeceraEspecial{\'Indice}

\tableofcontents

\newpage 

% \'Indice de figuras
%
% La idea es semejante que para  el .toc del \'Indice, pero ahora se usa
% extensión .lof (List Of Figures) con la información de las figuras.

\cabeceraEspecial{\'Indice de figuras}

\ifpdf
   \pdfbookmark[1]{\'Indice de figuras}{\'indice de figuras}
\fi

\listoffigures

\newpage

% \'Indice de tablas
% Como antes, pero ahora .lot (List Of Tables)

\ifpdf
   \pdfbookmark[1]{\'Indice de tablas}{\'indice de tablas}
\fi

\cabeceraEspecial{\'Indice de tablas}

\listoftables

\newpage

% Variable local para emacs, para  que encuentre el fichero maestro de
% compilación y funcionen mejor algunas teclas rápidas de AucTeX

%%%
%%% Local Variables:
%%% mode: latex
%%% TeX-master: "../Tesis.tex"
%%% End:

  \fi

  % Marcamos el  comienzo de  los capítulos (para  la numeración  de las páginas y ponemos la cabecera normal
  \mainmatter
  \restauraCabecera

  % \chapter{INTRODUCCI\'ON}

\section{ANTECEDENTES}
\label{cap1:sec:antecedente}




El comienzo de la psicrometría se remonta al año 1825, cuando Ernest Ferdinand August usa la palabra latina \textit{psychro} para nombrar su invento, un termómetro de bulbo húmedo \parencite[ p. 9]{Gatley2013}; sin embargo, existen autores que señalan, que la palabra \textit{psicrometría} proviene de las palabras griegas $\psi \upsilon \chi \rho \acute{o} \upsilon $ que significa frio, y la palabra $\mu \acute{\varepsilon} \tau  \rho o \upsilon$ que significa medida \parencite{Shallcross1997}. 

% (traducido del término inglés, coldness, acuñado en el siglo XVIII)

A principios del siglo IXX, recién se ha hace uso el término psicrometría para describir a un dispositivo muy semejante al que se conoce en la actualidad, un termómetro con bulbo húmedo y otro seco, arreglado de lado a lado. Hoy en día, el psicrómetro, como instrumento, se refiere a cualquier dispositivo que permite medir, por cualquier medio, el contenido de vapor en una mezcla gas-vapor; el caso más común de su uso es en la medida del contenido de humedad o vapor de agua en el aire seco \parencite{Shallcross1997}. Todo ello, fue posible gracias al invento del termómetro, el cual se atribuye a Galileo Galilei (1564-1642), en el siglo XVI.

En vista de las complejidades en el cálculo de las variables psicrométricas, se han desarrollado diagramas o cartas para realizar las lecturas de las propiedades de tales variables. 
El primer diagrama psicrométrico fue desarrollado por Willis Carrier, alrededor del año 1900 (Brandt, 2016); por aquel entonces, se  conocía como cartas higrométricas. En 1911 definitívamente se ha renombrado a diagramas o cartas psicrométricas \parencite[ p. 10]{Gatley2013}.% (Gatley, 2013, p. 10).

% En latinoamérica, en los años 1970 fue de uso común las tablas psicrométricas en el area de meteorología, muy similar a la tabla de logaritmos; tal como se evidencia en los archivos encontrados en las páginas del Instituto de Hidrología, Meteorología y Estudios Ambientales (antes conocido como, Servicio Colombiano de Meteorología e Hidrología). Estas tablas permiten determinar la humedad relativa, la tensión de vapor (en milibares, mb), el Punto de Rocío (en grados Celsius), tomando como variables de entrada la temperatura de bulbo seco y la temperatura de bulbo húmedo.
En latinoamérica, en los años 70, fue de uso común las tablas psicrométricas en el area de meteorología, muy similar a la tabla de logaritmos; tal como se evidencia en los archivos encontrados en las páginas del Instituto de Hidrología, Meteorología y Estudios Ambientales (antes conocido como, Servicio Colombiano de Meteorología e Hidrología). Estas tablas permiten determinar la humedad relativa, la tensión de vapor (en milibares, mbar), el Punto de Rocío (en grados Celsius), tomando como variables de entrada la temperatura de bulbo seco y la temperatura de bulbo húmedo. Actualmente las instituciones del estado boliviano, se usa unas reglas manuales para determinar la humedad relativa, llamadas regla psicrométrica.

La \glsxtrfull{ashrae}, quien desde 1984 ha desarrollado directrices para Sistemas de Calefacción, Ventilación, Acondicionamiento\footnote[1]{En las traducciones españolas erroneamente llamado, climatización. Ver en el anexo de definiciones.} y Refrigeración (HVAC $\&$ R); recién en 1996, introduce el término \textit{psicrometría} dentro de su glosario.

% En 1997, Shallcross, profesor de Ingeniería Mecánica en la Universidad de Melbourne, publica los principios para la construcción de los diagramas psicrométricos donde menciona que pueden ser generados a partir de las ecuaciones viriales para diversas mezclas de vapor-gas.


\subsection*{CONCEPTOS}

Principio de los estados correspondientes.
El estudio de la psicrometría se fundamenta en los principios de la termodinámica clásica.



% Son diversas las definiciones que se le han dado al estudio de la psicrometría; a continuación se mencionan algunos de ellos.
Son diversas los conceptos que se han manejado en el estudio de la psicrometría; a continuación se mencionan algunos de ellos.

Para \parencite[ p. 25]{SamuelC.Sugarman2007} la psicrometría es la ciencia de la termodinámica del aire húmedo aplicado a los sistemas \glsxtrshort{hvac}. 
% Enfriamento: Disminución de la temperature de aire.
% Teoría de procesos de humidifcación y deshumidificación


% \textcite{Rajput2007}, hace una distición entre los términos, psicrometría y psicrométrica; donde la psicrometría conceptualiza como el arte   

% Los diagramas que se han revisado 

% De acuerdo a  \href{www.wikipedia.org}{wikipedia} 
Para \textcite[]{Himmelblau2004}, un diagrama de humedad o carta psicrométrica, no es mas que una representación gráfica del balance de materia y energía de la mezcla vapor de agua y aire seco; evidententemente asociados a las variables psicrométricas.


Según \textcite[]{Martinez1992}, la psicrometría es el estudio del aire húmedo. Mientras que \textcite[]{YunusA.Cengel2015} evita usar el término psicrometría y solamente refiere como mezclas de gas-vapor y acondicionamiento de aire.    




% Para () señala que diversos autores definen la psicrometría como la ciencia que estudia las propiedades de la mezcla de aire (se entiende aire por aire seco) y vapor de agua; en el ámbito de procesos ()

Para la elaboración de este proyecto, se tomará la definición de \parencite[]{Perry2000}, donde la psicrometría se refiere a la determinación de las propiedades de la mezcla de gas-vapor, donde la mezcla aire seco-vapor de agua es el sistema más representativo.



% La psicrometría es una rama de la ciencias físicas que se ocupa de estudiar las propiedades termodinámicas de mezclas de gas-vapor; en particular, el aire húmedo; sin embargo, puede extenderse a mezclas de otros compuestos.

% Dentro del area de ingeniería (HVAC) su uso está bastante extendido, el cual se ocupa en dar soluciones de un entorno confortable el cual consiste en:




% se ha estudiado los griegos como la medición de frio. 

\subsection*{IMPLEMENTACIONES INFORMÁTICAS}

Las primeras implementaciones informáticas se han dado en el lenguaje de programación FORTRAN-70; desde entonces, se han desarrollado diversos programas para PC's de los cuales se pueden listar a continuación:
\begin{itemize}
    \item Psicro (2009),  desarrollado por \parencite{Compagnon2010}
    % \item Interfaz gráfica del diagrama psicrométrico Aplicado en problemas de humidificación y secado \parencite{Zamario1000}
    \item Simulación de los procesos psicrométricos utilizando el lenguaje de programación java \parencite{Cardoso2016}
    \item PsychroLib, es una biblioteca de funciones psicrométricas para calcular las propiedades termodinámicas del aire \parencite{Meyer2019}
    % \item Entre otras que se pueden encontran en diversos libros de transferencia de masa.
\end{itemize}

El trabajo más destacado de los mencionados previamente es de \textcite[]{Meyer2019} que, al no encontrar implementaciones numéricas apropiadas para los cálculos de las propiedades psicrométricas de manera libre, publica un conjunto de bibliotecas en el repositorio de software \href{https://github.com/psychrometrics/psychrolib}{Github}, para la mezcla de vapor de agua y aire seco; escrito para los lenguajes de programación, C/C++, C\#, Javascript, Python, R y Visual Basic para Aplicaciones (VBA); los cuales fueron publicados bajo la licencia \href{https://github.com/psychrometrics/psychrolib/blob/master/LICENSE.txt}{MIT}. %Sin embargo respecto a otros sistemas gas-vapor, no se encuentra mucha información.

Los softwares especializados en el area de procesos, tales como: ASPEN ONE, Chemcad y DWSIM, disponen de la capacidad para generar estas cartas o diagramas. Este conjunto de softwares unicamente generan imágenes estáticas; inmediatamente se percibe la falta dinamismo y capacidad de mostrar resultados de los resultados de manera interactiva. Estos softwares, también requiere de una PC para su instalación y funcionamiento y generar tales diagramas; por lo cual, se hace necesario el desarrollo de una \glsxtrfull{apk}\footnote{También llamado, aplicación.}  para este propósito. % \footcite{A partir de ahora unicamente, \glsxtrshort{apk}. %; a menos que se indique lo contrario.} para esta propósito. 






% \subsection*{Xxxxxxxxx}
% En nuestra sociedad, dada la dificultad de y un conocimiento empírico de los técnicos hace que acceder a un PC sea ligeramente complejo.

% Además de ello, en virtud del uso de las cartas psicrométricas (cada vez que ocurre una variación de temperatura, siempre) se necesita acceder a ello para realizar un nuevo cálculo.
Hoy en dia, la mayoría de las personas disponen de teléfonos inteligentes; por lo cual, es más que razonable disponer de una aplicación para el cálculos de las propiedades psicrométricas.

La ASHRAE recientemente ha publicado su aplicación del diagrama psicrométrico  del aire húmedo en la tienda AppStore, para telefonos IOS; mas no en la tienda PlayStore, para para teléfonos Android.

Todo los diagramas que se han revisado unicamente presentan diagramas para el aire húmedo. Los diagramas psicrométricos represntado mediante ecuaciones viriales, pueden extenderse para mezclas vapor-gas de otras sustancias.


\subsection*{APLICACIONES DE LA PSICROMETRÍA} 

Las aplicaciones de la psicrometría se extiende por diversas areas de la ingeniería; desde la térmica, meteorología, agricultura, hasta la industria textil.

En el area de agricultura, puede contribuir a reducir la pérdida de agua por evaporación durante el riego  por aspersión o goteo de los campos de cultivo; mejorando de esta manera el uso del agua en la agricultura.

También puede ayudar a desarrollar nuevas nuevas formas de cultivo en espacios reducidos; tales como el cultivo de hongos comestibles de la especie \textit{Suillus luteusy} en el altiplano, cuando este se cultiva generalmente en hectareas de pinos de la especie: \textit{Pinus radiata} o \textit{Pinus patula}.

En la meteorología, ayuda con el pronóstico del tiempo, en cuanto a las precipitaciones, o el grado de saturación de la humedad en la atmósfera.

Un diagrama psicrométrico para el vapor de $\ch{CO2}$ y aire seco, puede ayudar a estudiara y desarrollar nuevas formas reducción del $\ch{CO2}$ de la atmosfera.


% la búsqueda de diagramas implica tiempo
% Conforme el avance de la tecnología; también se hace necesario


% Se ha implementado en diversas plataformas para PC y teléfonos móviles; sin embargo, ninguna de ellas cuenta con con la flexibilidad que se desea para realizar las lecturas sin recurrir a lecturas erroneas. En la tienda PlayStore para telefonos android, se puede encontrar diversas aplicaciones informáticas, pero únicamente para la mezcla de aire húmedo.

% Una forma simple de mostrar
% Una carta psicrométrica es una representación gráfica de las propiedades termodinámicas, generalmente de la mezcla aire-vapor de agua

% Diferencias entre gas y vapor.
% El vapor puede condensarse
% El gas permanece 

% Diagrama carta o índice

% Diseño de los Diagramas psicrométricos

% Los primeros algoritmos para calcular las variables psicrométricas fueron escritos en el lenguaje de programación FORTRAN 77 (1978)






% puede que se vaya a la seccion de estado del arte
% la disponibilidad de diagramas psicrométricos para mezclas no convencionales es bastante limitado.




% En la actualidad es una ciencia completamente desarrollada, alejado de su pasado empírico

% Diversos autores han sintetizado la representación de las propiedades 

\section{PLANTEAMIENTO DEL PROBLEMA}

\section{DEFINICIONES}
\label{cap1:sec:definiciones}



% \printglossaries % solo trabala con \package{glosaries}
\printunsrtglossary

% Revisar esto para introducir acrónimos y glosarios 
% https://tex.stackexchange.com/questions/8946/how-to-combine-acronym-and-glossary
% 
\section{OBJETIVOS}
\label{cap1:sec:objetivos}

\subsection{OBJETIVO GENERAL}
\begin{itemize}
    \item Desarrollar una aplicación para la plataforma Android que permita visualizar cartas psicrométricas para múltiples mezclas vapor del líquido y gas seco, relacionada con aplicaciones ambientales. %en un solo punto. %para más de una mezcla gas-vapor.
\end{itemize}
\subsection{OBJETIVOS ESPECÍFICOS}
    \begin{itemize}
        % \item Desarrollar una aplicación para múltiples mezclas vapor-gas la información que permita construir las cartas psicrometétricas hospedados en la nube en base a una combinación de variables.
        % \item La variable principal será presión atmosférica.
        % \item Proveer a los técnicos en el area de ingeniería de una app facil de leer las variables psicrométricas
        % \item Publicar el código fuente bajo la licencia Apache v2.0
        % \item Atenuar los colores de las lineas para una mejor lectura
         
        % \item Desarrollar una aplicación de forma que permita la visualización del diagrama psicrométrico para mezclas diferentes de vapor-gas.
        \item Desarrollar la aplicación en el lenguage de programación \ldp, de manera que permita la extensión a mezclas diferentes vapor del líquido - gas seco.
        \item Implementar para la mezcla vapor de agua-aire seco, de manera predeterminada.
        % \item Implementar para la mezcla de vapor de $\ch{CO2}$ y aire seco.
        % \item Implementar el diagrama psicrométrico para la mezcla, Benceno - Nitrogenó.
        \item Desarrollar algoritmos para la determinación de las variables psicrométricas dependientes: Humedad absoluta, Humedad relativa, temperatura de bulbo húmedo, temperatura de rocío, volumen específico. 
        
        % \item Mostrar los resultados obtenidos de las variables psicrométricas en un cuadro de visualización para su lectura inmediata.
        % \item Implementar para el sistema de unidad internacional, imperial y personalizadas
        


        %adicionar despues XXX
        % \item Implementar la aplicación para el \acrfull{si} de unidades y otras unidades de interés en las ciencias ambientales. % y personalizadas. % selección dinámica de unidades
        

        % \item Implementar para el \acrfull{si} de unidades y unidades convencionales relacionadas al area de meteorología y climatología. % y personalizadas. % selección dinámica de unidades
        \item Comparar los resultados obtenidos a partir de la aplicación desarrollada con el simulador \href{https://dwsim.fossee.in}{DWSIM}.
        \item Justificar el uso de la ecuacion de gases ideales respecto a la ecuación virial de gases para las mezclas de vapor de agua $-$ aire seco. %, a una presión definida.

        \item Presentar a Google para su comprobación y aprobación de la originalidad del  documento terminado.
        \item Mostrar el producto final en la tienda oficial de aplicaciones de Google, PlayStore.
        \item Desarrollar dos casos de aplicación de la herramienta en el area de Ingeniería Ambiental 
        % \item Implementar 
        % \item Escribir código por lo menos para 3 mezclas binarias gas-vapor.
        % \item Mostrar dinámicamente las coordenadas
        % \item Implementar el diagrama psicrométrico para una mezcla extra vapor-gas.
        % \item Comparar los resultados obtenidos con la app, la regla psicrométrica que utilizan las entidades del estado, tales como \acrfull{senamhi} y una algoritmo externo.
        % \item Comparar la diferencia en el aprendizaje de tópicos relacionados a psicrometría con la aplicación y sin ella. %entre los resultados 
        % \item Identificar las razones del uso de reglas  manuales psicrométricas dentro del \acrfull{senamhi}, en base a ello, realizar sugerencias a dicha institución.
        
        % \item Sugererir a la \acrfull{rae}, la corrección del significado dado a la palabra, \textit{climatizar}.
        % \item Comparar los resultados obtenidos con la app, la regla psicrométrica que utilizan las entidades del estado, tales como \acrfull{senamhi} y una algoritmo implementado externo
        % \item Disponer de serie de diagramas para la mezcla de gas-vapor en un solo punto y acceder a ellas desde cualquier punto (disponible para cualquiera que desee consultar),
        % \item buscador, numero de búsquedas, tiempo que les a tomado, tipo de pc, provedor de internet, calidad del producto encontrado (tiempo de apendizaje con la app y sin app, formato de la imagen pdf, jpeg. fuente de obtencion, elaboracion, link, sistema operativo)
   
    \end{itemize}

% \section{Justificacion}
% \label{cap1:sec:justif}



  \chapter{MARCO TE\'ORICO}
\label{cap2}

\begin{FraseCelebre}
  \begin{Frase}
    La mejor estructura no garantizará los resultados ni el rendimiento.
    Pero la estructura equivocada es una garantía de fracaso.
  \end{Frase}
\begin{Fuente}
Peter Drucker
\end{Fuente}
\end{FraseCelebre}

\begin{resumen}
  Este capítulo explica la estructura de directorios de extendido posteriormente en el
  % capítulo~\ref{cap:makefile}.
\end{resumen}

% \section{El concepto de Estado en Termodinámica}

%%% OK
\section{ESTADOS DE LA MATERIA}
La termodinámica clásica se encarga de estudiar los estados de agregación: sólido, líquido y gaseoso. 
La fase gaseosa, donde las moléculas están bastante apartadas y sin orden aparente, adopta la forma del recipiente que los contiene \parencite[p. 111]{YunusA.Cengel2015}; 
en la fase líquida, las fuerzas intermoleculares son más débiles y también adopta la forma del recipiente  que los contiene. Para \parencite{ReyesChumacero2012}, las principales diferencias de los distintos estados de la materia están en el arreglo dinámico de sus partículas (posición, distancia y movimiento).

\subsection{ESTADO TERMODINÁMICO}
Un estado termodinámico queda definido por las \textit{propiedades termodinámicas} \parencite[26]{Smith.VanNess.Abbott2007}. Son  propiedades termodinámicas la presión, temperatura y el volumen; estas propiedades depende unos de otros.
\newline

Las propiedades pueden clasificarse en extensivas e intensivas. \newline
Las \textit{propiedades extensivas} son todas aquellas que depende de la cantidad de sustancia; es decir, su valor cambia con la cantidad de sustancia .
Las \textit{propiedades intensivas}, independiente de la cantidad de sustancia, este es invariante.

\subsection{FUNCIONES DE ESTADO}
Las propiedades termodinámicas 


\subsection{VARIABLES DE ESTADO}
Las variables de estado definen el estado de un sistema termodinámico. Para \parencite{castellan1998fisicoquimica}, el estado se define mediante las siguientes propiedades: masa, volumen, presión y temperatura. Mediante una { \it ecuación de estado}, puede determinarse la cuarta propiedad, el cual se conoce a partir del comportamiento del sistema.

% \subsection{Propiedades o Variables Termodinámicas}
% El estado termodinámico queda establecido mediante las propiedades termodinámicas como la termperatura, presión y densidad.
% Las cantidades intensivas que no depende a los sucesos que se haya sometido el sistema termodinámico; estas propiedades termodinámicas se denominan {\it funciones de estado} \parencite{Smith.VanNess.Abbott2007}


% antes cambio de estado: No se puede hablar de equilibrio sin antes definir el concepto de estado termodinámico
\subsection{EQUILIBRIO}
Según \parencite[p. 29]{Smith.VanNess.Abbott2007}, es la ausencia de cambio, lo cual denota una condición estática; bajo esta condición no puede ocurrir ningún cambio de estado e indica la ausencia de fuerzas impulsoras hacia un cambio de estado.

\subsection{EQUILIBRIO TERMODINÁMICO}
El equilibrio se como el estado de balance donde las fuerzas se cancelan mutuamente. Existen diferentes tipos de fuerzas, de los cuales se mencionan a continuación:
\begin{itemize}
  \item Equilibrio mecánico,
  \item equilibrio térmico,
  \item equilibrio químico,
  \item equilibrio de fases,
  \item equilibrio termodinámico.
\end{itemize}

Para \parencite{Reisel2021} y \parencite{YunusA.Cengel2015}, el equilibrio termodinámico sucede cuando un sistema alcanza el equilibrio mecánico, térmico, químico y el equilibrio de fases.
Un sistema en equilibrio termodinámico es incapaz de realizar un cambio espontaneo de su estado; cualquier cambio de estado recibe el nombre de proceso \parencite[p. 39]{YunusA.Cengel2015}.




\subsubsection{FASE}
    De acuerdo a \parencite[p. 30]{Smith.VanNess.Abbott2007}, una fase es una región homogénea de la materia; es decir, un gas, una mezcla de gases, un líquido, una solución de líquidos, un sólido cristalino son algunos ejemplos de fases. Para \parencite[p. 461]{ChangQmc2010}, una fase es una parte homogénea de un sistema, y aunque está en contacto con otras partes del sistema, está separada de esas partes por un límite bien definido.
    
    Pueden coexistir varias fases con la condición de que deben estar en equilibrio. A los líquidos y sólidos también se les conoce como fases condensadas o (estados condensados) \parencite[p. 430]{SilberberglQmcGeneral}. 

\subsection{CAMBIOS DE FASE}
    También conocido como transformaciones de fase, ocurre cuando se agrega o quita energía, generalmente en forma de calor. Los cambios de fase son cambios físicos que se distinguen porque  cambia el orden molecular, y en la fase gaseosa las moléculas presentan un mayor desorden \parencite[p. 489]{ChangQmc2010}. Los cambios entre los diferentes estados de agregación, según \parencite{ReyesChumacero2012}, se debe a las interacciones a nivel atómico y molecular de las sustancias y se manifiestan a nivel macroscópico como fuerza de {\it cohesión}, el cual sucede entre moléculas idénticas y las fuerzas de {\it adhesión} que se dan entre moléculas diferentes.

     
  % \figPhaseDiagram

De la figura \ref{fig:diag_phase}, la temperatura de equilibrio para la coexistencia de la fase Sólida-Líquida a una presión dada se llama, temperatura de fusión normal o temperatura de congelación normal; para la temperatura de equilibrio a una determinada presión entre la fase Líquido-Gas se llama {\it Temperatura de ebullición normal}. "Si el punto triple está a una presión superior a 1 , la sustancia no tiene una temperatura de congelación normal o una temperatura de ebullición normal, pero tiene una temperatura de sublimación normal a la que el sólido y el gas coexisten a una presión igual a 1 atmósfera", \parencite[pp. 28]{Mortimer2008} %47   

\subsection{Estructura Cristalina}
    La fase sólida se divide en dos categorías:
    % \begin{enumerate }
    %   \item sólidos cristalinos 
    %   % (las fuerzas que mantienen su estabilidad pueden ser iónicas,  covalentes, de van der Waals, enlaces de hidrógeno o una combinación de todas ellas.)
    %   \item sólidos amorfos (ej. vidrio)
    % \end{enumerate}

    \subsubsection*{Tipos de cristales}
    \begin{itemize}
      \item cristales iónicos (formado por cationes y aniones que suelen ser de distinto tamaño)
      \item cristales covalentes (se mantienen en una red tridimensional. Ej: diamante y grafito)
      \item cristales metálicos (los electrones están deslocalizado en todo el cristal)
    \end{itemize}


\subsection{Equilibrio Líquido-Vapor}
    

\subsubsection*{Tipos de sólidos amorfos}
    Se forma cuando los átomos y moléculas del fluido no tienen tiempo de alinearse por sí mismos y pueden quedar fijas en posiciones distintas a uno de cristal ordenado
    \begin{itemize}
      \item cristales iónicos (formado por cationes y aniones que suelen ser de distinto tamaño)
      \item cristales covalentes (se mantienen en una red tridimensional. Ej: diamante y grafito)
      \item cristales metálicos (los electrones están deslocalizado en todo el cristal)
    \end{itemize}


    Cuando las moléculas de un líquido tienen suficiente energía para escapar de la superficie, sucede un cambio de fase

\subsection{Presión de vapor}
    Cuando algunas moléculas de un líquido tienen energía suficiente para vender las fuerzas atractivas, y pasar al estado gaseoso, el proceso se llama evaporación o vaporización (Universidad de alcalá, cap 8)


\subsection{Mezclas Gas-Vapor}
Las mezclas debajo de la temperatura crítica


\subsection{Propiedades de los Fluidos}


\subsubsection{Regla de Fase}
    \begin{equation}
    \label{eq:thermo:phase_rule}
      \eqPhaseRule
    \end{equation}

    Donde,\newline 
    \freeDegree : es el número de grados de libertad o las variables que se tienen que especificar, \newline 
    \componentNumber : representa al número de componentes y \newline
    \phaseNumber : es el número de fases (sólido, líquido o gaseoso) en el sistema.
    % \includesvg[width=4.0in]{Imagenes/Vectorial/thermo/coord_P_V.svg}

  \figCoordinatePV

\subsubsection{Naturaleza de las mezclas Gas-Vapor}
el
\subsubsection{Naturaleza del Equilibrio}

\subsubsection{Equilibrio Dinámico}
\subsubsection{Sistema Termodinámico}

\subsubsection*{Modelo para el Equilibrio Líquido-Vapor}
\subsubsection*{Variables Macroscópicas Volumen, Presión, Temperatura}

\subsubsection{Temperatura}
En la actualidad se define en función a la constante de Boltzmann\footnote{Generado mediante \href{https://docs.scipy.org/doc/scipy/reference/constants.html}{cipy} y pythontex} {$k$}, \pcu{Boltzmann constant} en
\subsubsection{Presión}
En general, la presión se define como la fuerza normal ejercida sobre una unidad de area. Se habla de presión cuando el sistema consiste de fluidos (gas o líquido); pero cuando este es sólido, se usa el término \textit{esfuerzo normal} \textcite{YunusA.Cengel2015}

\begin{equation}
  \eqPressure
\end{equation}

\figMolecularColissions

de un sistema en fase gaseosa, es la presión que aporta uno de los componentes


\subsubsection{Presión Parcial}
\subsubsection{Presión Parcial}

\subsubsection{Presión Parcial}
como la fracción de uno de los componentes respecto a la suma de cada una de ellas, denominado presión total.

\eqGasCompositionFromPressure
Donde \symCompGV{i} es la composición molar, {\P{i}} es la presión del componente {${i}$}.

la expresión \eqref{eq:thermo:GasCompositionFromPressure}, también puede expresarse de la siguiente forma:

\eqGasCompositionFromTotalPressure
Donde

\subsubsection*{Ley de Raoult}
Se requiere hacer las siguientes suposiciones para las consideraciones para la ley de Raoult \parencite{Smith.VanNess.Abbott2007}
\begin{itemize}
  \item La fase de vapor es un \textbf{gas ideal}; significa que la aplicación se aplica solo a presiones bajas y moderadas.
  \item la fase líquida es una \textbf{solución ideal}; implica que la validéz aproximada se da cuando las especies que constituyen el sistema, sean químicamente semejantes\footnote{Especies moleculares de similar tamaño y de la misma naturaleza química.}
\end{itemize}

En base a las suposiciones mencionadas se deduce la siguiente expresión matemática

\begin{equation}
  \eqRaoultLaw
\end{equation}

Donde {$\symCompGV{}_{i}$} es la composición molar de la fase vapor,  {\pressure{}{}} es la presión total de la mezcla, {$\symCompL{}_{i}$} representa a la composición molar de la fase líquida del sistema y

% obs
De la ecuación \eqref{eq:thermo:RaoultLaw}, la expresión del lado izquierdo se conoce como, presión parcial del componente ${i}$
% El gas ideal permite la comparación del comportamiento del gas real 


\begin{align}
  f_{i}^{\degree, \text{ fase vap}} & = \hat{\phi_{i}} \symCompGV{i} \pressure{}{} \\
  f_{i}^{\degree, \text{ fase liq}} & = \hat{\phi_{i}} \symCompL{i} p_{i}
\end{align}

\begin{flalign}
  f_{i}^{\degree, \text{ fase vap}} & = \hat{\phi_{i}} \symCompGV{i} \pressure{}{} \\
  f_{i}^{\degree, \text{ fase liq}} & = \hat{\phi_{i}} \symCompL{i} p_{i}
\end{flalign}

\begin{table}[htbp]
  \centering
  \caption{prueba}
  \rowcolors{1}{gray!20}{}
  \begin{tabular}{ccc}
    \hline
    \rowcolor{color_aquamarine} Variable & En la Ley de Raoult                              & En la Ley de Raoult modificada                               \\
    \hline
    $\hat{\phi_{i}}$                     & 1.00                                             & 1.00                                                         \\
    \symCompGV{i}                        & \symCompGV{i}                                    & \symCompGV{i}                                                \\
    $f_{i}^{\degree \text{, fase vap}}$  & \pressure{}{}                                             & \pressure{}{}                                                         \\
    ${\gamma}_{i}$                       & 1.00                                             & ${{\gamma}_{i}}$                                             \\
    $f_{i}^{\degree, \text{ fase liq}}$  & $p_{i}$                                          & $p_{i}$                                                      \\
    Ec. resultante                       & $\symCompGV{i} = \frac{\symCompL{i} {P_{i}}}{P}$ & $\symCompGV{i} = \frac{ \gamma_{i} \symCompL{i} {P_{i}}}{P}$ \\ \hline
  \end{tabular}
  % Tomado de: PHYSICAL AND CHEMICAL EQUILIBRIUM FOR CHEMICAL ENGINEERS

\end{table}


\subsubsection*{Ley De Henry}

%-------------------------------------------------------------------
Según \parencite{Engel2019}, una limitación de la ley de los gases ideales es que no predice bajo las condiciones apropiadas la licuefacción\footnote{cambio de estado de una sustancia cuando pasa del estado gaseoso al líquido} de gases.



\section{ECUACIONES DE ESTADO}
    Los estudios realizados por R. Boyle(1662), Charles () y Gay Lussac, a partir de sus estudios independientes, permitieron la generalización del comportamiento los gases mediante una relación matemática conocida como {\it ley de gas ideal}. 

\subsection{ECUACIÓN DE GAS IDEAL}
    El comportamiento ideal de un gas, se expresa mediante la siguiente ecuación de estado:

    \begin{equation}
      \eqIdealGas
    \end{equation}

    Donde {\zFactor} es el coeficiente de compresibilidad, {\pressure{}{}} es la presión absoluta, {\Vmolar} es el volumen molar, {\T{}} temperatura y {\univConstGasEos} es la constante universal de los gases.

\subsection{ECUACIÓN VIRIAL GENERAL}
    \label{cap2:subsec:tagEqnvirial}
    Para \parencite[p. 37]{Sengers1987equations}, la ecuación virial de estado de los gases, es una de las ecuaciones que intenta describir de manera exacta y precisa el comportamiento de los fluidos reales. 

    Una ventaja del uso de la ecuación virial es la existencia amplia de información experimental \Parencite[p. 38]{Sengers1987equations}, 

    La ecuación virial puede ser apreciada como una serie de Maclaurin\footnotetext{También llamado serie de Taylor}, donde ${\pressure{}{}}/{\univConstGasEos \T{}}$; en vista de su naturaleza polinomial, puede escribirse como serie de potencias de $1/\Vmolar$, el cual se muestra a continuación:

    \begin{equation}
      \eqVirialForVDevelop{}
    \end{equation}
    \eqVirialForVDevelopNaming
    
    El 2do y el 3er coeficiente virial ha sido estudiado por más de 100 años, todos ellos muestran que no existe correlación \parencite{thirdVirialTing}


    % Se considera las siguientes suposiciones:

  
\subsubsection*{Dependencia de los coeficientes viriales de la temperatura}
    \begin{equation}
      \bCoeffDependentOfT{}
    \end{equation}
    
    \begin{equation}
      \cCoeffDependentOfT{}
    \end{equation}

\subsubsection*{Dependencia de los coeficientes viriales de la composición}
    Para un gas puro, de acuerdo a la ecuación \eqref{eq:thermo:phase_rule} se necesita, dos grados de libertad (\freeDegree),  

    Donde, $x = 1$ 

\subsubsection{Métodos para la estimación de los coeficientes viriales}
    Diversos autores señalan que los coeficientes viriales pueden derivarse de \parencite[p. 4.13]{Poling2001} de la teoría molecular
    Para \parencite{Trusler10.1039/9781782627043-00152}, los métodos para la medición de los coeficientes viriales pueden clasificarse en:

    \begin{enumerate}[label=(\alph*)]
      \item Métodos directos
        \begin{itemize}
          \item Aparato de Burnet
          \item Densímetro de doble plomada acoplados magnéticamente
        \end{itemize}
      \item  Métodos indirectos
        \begin{itemize}
          \item Flujo calorimétrico
          \item Velocidad del sonido
        \end{itemize}
    \end{enumerate}

    para \parencite{Dymond2002}, los métodos para la determinación de los coeficientes viriales pueden clasificarse de la siguiente manera:

    \begin{itemize}
      \item Medidas de {\pressure{}{}-\Vmolar-\T{}},
      \item Medidas de la velocidad del sonido
      \item Medidas de Joule-Thomson
      \item Medidas de índice de refractividad y permitividad relativa
      \item Medición de la entalpía de vaporización y la presión de vapor. 
    \end{itemize}

\subsubsection{Estimación del Segundo Coeficiente Virial}
  
    El segundo coeficiente virial representa a la interacción entre pares de moléculas

%%%%
Son diversos los trabajos que se han realizado para determinar el segundo coeficiente virial. \autocite[postnote]{woolley1969second} trabajó para la formulación analítica del segundo coeficiente virial para una función esféricamente simétrica al cual denominó, \textit{potencial de par realista}. \parencite{pitzer1990second} también trabajó para estimar el segundo coeficientes virial para compuestos no polares y de baja polaridad. Para compuestos polares se han realizado diversos estudios para estimar el segundo coeficiente virial, entre los cuales destacan los estudios de \parencite{maris1985interaction}, \parencite{tarakad1977improved}, y entre otros estudios se han realizado . \parencite{vetere2007simple} propuso una modificacion a método de Pitzer para predecir el segundo coeficiente virial para compuestos puros.

\parencite{klotz1985improved} presenta una mejora para el cálculo del 2do coeff virial



Para la temperatura $\T{}$ y composición molar $\symCompGV{}$, ambos constantes; de la ecuación \eqref{eq:virialB}, cuando $1 / \Vmolar\ \to 0$,  el segundo coeficiente virial $\bCoeffVirial{}$, puede escribirse como:

\begin{equation}
  \bCoeffVirial{} = \lim_{\frac{1}{\Vmolar}\ \to {0}} \Vmolar \left(\frac{\pressure{}{} \Vmolar} {\univConstGasEos \T{}} - 1 \right)
\end{equation}

Para \parencite[p. 473]{beattie1942second}, la expresión $\Vmolar \left(\frac{\pressure{}{} \Vmolar} {\univConstGasEos \T{}} - 1 \right)$, puede ser calculado a partir de las mediciones de la compresibilidad isotérmica de un gas puro; en caso de una mezcla, debe ser de composición molar constante; la compresibilidad isotérmica debe establecerse frente a $1 / \V$. El valor de la constante queda establecido en la intersección de la función con el eje $1/ \V$.

Para una mezcla binaria, $\bCoeffVirial{}$ está relacionado con los coeficientes viriales de los componentes puros y sus composiciones, de la siguiente manera:

\begin{equation}
  \eqSecondVirialCoeffSum{}
\end{equation}

Para un sistema de dos componentes ($\symNumOfCmpnts = 2$),
\begin{equation}
  \eqSecondVirialCoeffDevelop{}
\end{equation}
\eqSecondVirialCoeffNaming{}

Conociendo los coeficientes viriales para los componentes puros, hace falta establecer las relaciones matemáticas para {\bCoeffVirial{\nameG \nameV}}.



\subsubsection{Estimación del Tercer Coeficiente Virial}
Para \parencite{Dymond2002}, los valores son preferentemente determinados mediante la información de compresibilidad de gases 

\begin{equation}
  \cCoeffVirial{} = \lim_{\frac{1}{\Vmolar}\ \to {0}} \left[ \Vmolar \left(\frac{\pressure{}{} \Vmolar} {\univConstGasEos \T{}} - 1 \right) - \bCoeffVirial{} \right] \Vmolar
\end{equation}


\begin{equation}
  \eqThirdVirialCoeffSum{}
\end{equation}

Desarrollando para un sistema de dos componentes:

\begin{equation}
  \thirdVirialCoeffDevelop{}
\end{equation}

\eqThirdVirialCoeffNaming{}

\figVirialBCBehavior

\begin{align}
  \eqVirialFactorCorrection{}
\end{align}

\factorCorrectionNaming{}

\subsection{Convergencia de la serie virial}

\subsection{Teoría del Sonido}
\subsubsection*{Velocidad del Sonido en los Gases}

\parencite[p. 186]{Wilhelm2010}, define la velocidad del sonido como:

\begin{align}
  \label{cap:subsec:velocidadSonido}
  \SoundSpeed{}
\end{align}

Donde $\pressure{}{}$  es la presión, $\Density$ es la densidad, $\Entropy$ es la entropía

\subsection{Correlación, predicción y estimación de los coeficientes viriales}
\label{cap2:subsec:tagEqnPredict}




Según \parencite[ p. 12]{Dymond2002}, el método basado en la teoría del sonido, es el método que mejor se acerca a los valores reales; de esta manera, el segundo coeficiente virial queda expresado de la siguiente forma:
\begin{equation}
  \eqSndVirialCoeff{1}{2}{\T{}}
\end{equation}


Donde $f_{12}$ = $ \ce{exp}{\{-U(R, \omega_1 , \omega_2)/kt\}} - 1 $. Para una molécula lineal:

Desarrollo de las ecuaciones para el cálculo de los coeficientes viriales fluidos polares y no polares





\subsection{VOLUMEN MOLAR PARA UNA MEZCLA}
    Para una mezcla el volumen molar
  \begin{equation}
    \eqVirialForVDevelopForMixture
  \end{equation}
    
    \eqVirialForVDevelopMixtureNaming

\section{PSICROMETR\'IA}

Para el estudio de la psicrometr\'ia mediante la ecuaci\'on de gases ideales, seg\'un (), se hacen las siguientes suposici\'ones ():

\begin{itemize}
  \item La fase s\'olida o l\'iquida no contiene gases disueltos,
  \item la fase gaseosa puede ser tratado como una mezcla de gases ideales,
  \item cuando la mezcla y la fase condensada est\'an dadas a una presi\'on y temperatura, el equilibrio entre la fase condensada y la mezcla gaseosa, no est\'a influenciada por la presencia de otros componentes; significa que cuando se alcanza el equilibrio, la presi\'on parcial del vapor ser\'a igual a la presi\'on de saturaci\'on dada a la temperatura de la mezcla.
\end{itemize}

La relaci\'on psicrométrica viene dada por:
\begin{equation}
  \frac{h_{c}}{k_{c_s}^{'}} = r
\end{equation}

\begin{equation}
  \eqVanDerWaals{p}{\overline V}{\theta}
\end{equation}

\subsection{Relación de humedad o humedad absoluta}
\begin{align}
  \eqHumidityRatioFst{\HumidityRatio}
\end{align}

\begin{align}
  \label{eq:psy:HumidityRatio2form}
  \eqHumidityRatioSnd{\HumidityRatio}
\end{align}

Si se conoce la humedad absoluta $\HumidityRatio$, la composición de la cantidad de vapor puede calcularse a partir de la ecuación \eqref{eq:psy:HumidityRatio2form}, quedando la expresión de la siguiente manera:
\begin{equation}
  \eqCompVapor{}
\end{equation}

\subsection{Humedad Relativa}
Son diversas las definiciones que se ha establecido; la definición termodinámica de acuerdo a (), es la relación de la fugacidad del vapor en la mezcla respecto a la fugacidad de la mezcla gaseosa.

Se define como la relación de la presión parcial de vapor del agua respecto a la presión saturada a una temperatura dada.

\begin{equation}
  \eqRelativeHumidityFst{\RelativeHumidity}
\end{equation}

\subsection{Punto de Rocío}
\subsection{Temperatura de Bulbo Húmedo}
\subsection{Temperatura de Bulbo Seco}



\subsection{Retos y deventajas}
Para \parencite[p. 8]{Bell2017}, puede ocurrir la formación potencial de hidratos de gas seco; estos son, moléculas de gas atrapados en una red de moléculas de vapor; este fenómeno suele suceder a xxx presiónes.
\subsection{Otras ecuaciones}











  \chapter{MARCO PRÁCTICO}
\label{cap3}

% 
% Nuevo en la versión 5.0 (2013)
% Incompatible con versiones antiguas
\begin{algorithm}[H] % H = forzar está posición
    \caption{}\label{Alg:AlgorithmELV}
    \SetAlgoLined
    \Datos{$presion, modelo \leftarrow 0, compuesto1,  compuesto2$}
    leerPropCritics(presion, compuesto1,  compuesto2)\\
    validarCompuesto(compuesto1,  compuesto2)\\
    $ {c1, c2} \leftarrow seleccionarCompuesto(compuesto1,  compuesto2) $\\
    ${isValid} \leftarrow validarMezcla()$
    \Resultado{$H, \V$}
    \LinesNumbered
    \SetAlgoVlined
    $presion \leftarrow 101.325$\;
    \eIf{modelo == 0}{
        $z \leftarrow 1$\;
        applyIdealModel(args)
    }{
        {readVirialCoeffs(compuesto)}\\
        {calcMixedVirialCoeffs(cCoeff1, cCoeff2)}\\
        $z \leftarrow \zVirial{\Vm} $\;
        applyVirialModel(model, vars)
    }
    \Return PsychrometicsVariables;
\end{algorithm}



\begin{algorithm}[H] % H = forzar está posición
    \caption{}\label{Alg:Psychrometry}
    \SetAlgoLined
    \Datos{$presion, modelo \leftarrow 0 $}
    \Resultado{$H, \V$}
    \LinesNumbered
    \SetAlgoVlined
    $presion \leftarrow 101.325$\;
    \eIf{modelo == 0}{

        $z \leftarrow 1$\;
        applyIdealModel(args)
    }{
        $z \leftarrow \zVirial{\Vm} $\;
        applyVirialModel(model, vars)
    }
    \Return PsychrometicsVariables;
\end{algorithm}



\Fn(\tcc*[h]{algorithm as a recursive function}) {\FRecurs{some args}}{
    \Datos{Some input data\\these inputs can be displayed on several lines and one
    input can be wider than line’s width.}
    \KwResult{Same for output data}
    \tcc{this is a comment to tell you that we will now really start code}
    \If(\tcc*[h]{a simple if but with a comment on the same line}){this is true}{
    we do that, else nothing\;
    \tcc{we will include other if so you can see this is possible}
    \eIf{we agree that}{
    we do that\;
    }{
    else we will do a more complicated if using else if\;
    \uIf{this first condition is true}{
    we do that\;
    }
    \uElseIf{this other condition is true}{
    this is done\tcc*[r]{else if}
    }
    \Else{
    in other case, we do this\tcc*[r]{else}
    }
    }
    }
    \tcc{now loops}
    \For{\forcond}{
    a for loop\;
    }
    \While{$i<n$}{
    a while loop including a repeat--until loop\;
    \Repeat{this end condition}{
    do this things\;
    }
    }
    They are many other possibilities and customization possible that you have to
    discover by reading the documentation.
}


  % 
\RequirePackage{etoolbox}
\RequirePackage{xparse}
\RequirePackage{siunitx}

\makeatletter

\def\physicalconstants@missingconstantmarker#1{%
    \ifx\@onlypreamble\@notprerr% only insert the marker when not in preamble
        {\textbf{??#1??}}%
    \fi
}
\def\physicalconstants@declare#1#2#3#4#5{% {name}{value}{unit}{uncertainty}
    \csdef{physicalconstants@#1@value}{#2}%
    \csdef{physicalconstants@#1@unit}{#3}%
    \csdef{physicalconstants@#1@uncertainty}{#4}%
    \csdef{physicalconstants@#1@uncertainvalue}{#5}%
}
\def\physicalconstants@blind@get#1#2{% {entry}{name}
    \csuse{physicalconstants@#2@#1}%
}
\def\physicalconstants@try#1{% {name}{subject}, gobbles subject if name is undefined
    \ifcsdef{physicalconstants@#1@value}{%
        \@firstofone
    }{%
        \GenericWarning{}{LaTeX Warning: I do not know the physical constant `#1'.}%
        \physicalconstants@missingconstantmarker{#1}%
        \@gobble
    }%
}
\def\physicalconstants@declarealias#1#2{% {new}{old}
    \physicalconstants@try{#2}{%
        \csdef{physicalconstants@#1@value}{\physicalconstants@blind@get{value}{#2}}%
        \csdef{physicalconstants@#1@unit}{\physicalconstants@blind@get{unit}{#2}}%
        \csdef{physicalconstants@#1@uncertainty}{\physicalconstants@blind@get{uncertainty}{#2}}%
        \csdef{physicalconstants@#1@uncertainvalue}{\physicalconstants@blind@get{uncertainvalue}{#2}}%
    }%
}
\def\physicalconstants@get#1#2{% {entry}{name}
    \physicalconstants@try{#2}{%
        \physicalconstants@blind@get{#1}{#2}%
    }%
}
\NewDocumentCommand\pcalias{}{% {new}{old}}
    \physicalconstants@declarealias
}
\NewDocumentCommand\pcget{}{% {entry}{name}
    \physicalconstants@get
}
\NewDocumentCommand\pcvalue{O{} m}{% [\num options]{name}
    \physicalconstants@try{#2}{%
        \num[#1]{\physicalconstants@blind@get{value}{#2}}%
    }%
}
\NewDocumentCommand\pcunit{O{} m}{% [\si options]{name}
    \physicalconstants@try{#2}{%
        \si[#1]{\physicalconstants@blind@get{unit}{#2}}%
    }%
}
\NewDocumentCommand\pcuncertainty{O{} m}{% [\num options]{name}
    \physicalconstants@try{#2}{%
        \num[#1]{\physicalconstants@blind@get{uncertainty}{#2}}%
    }%
}
\NewDocumentCommand\pcuncertainvalue{O{} m}{% [\num options]{name}
    \physicalconstants@try{#2}{%
        \num[#1]{\physicalconstants@blind@get{uncertainvalue}{#2}}%
    }%
}
\NewDocumentCommand\pc{O{} m}{% [\SI options]{name}
    \physicalconstants@try{#2}{%
        \SI[#1]{\physicalconstants@blind@get{value}{#2}}{\physicalconstants@blind@get{unit}{#2}}%
    }%
}
\NewDocumentCommand\pcu{O{} m}{% [\num options]{name}
    \physicalconstants@try{#2}{%
        \SI[#1]{\physicalconstants@blind@get{uncertainvalue}{#2}}{\physicalconstants@blind@get{unit}{#2}}%
    }%
}

\physicalconstants@declare{Wien displacement law constant}{0.0028977685}{\metre\kelvin}{5.1e-09}{0.0028977685(51)}
\physicalconstants@declare{atomic unit of 1st hyperpolarizablity}{3.20636151e-53}{\cube\coulomb\cube\metre\per\square\joule}{2.8e-60}{3.20636151(28)e-53}
\physicalconstants@declare{atomic unit of 2nd hyperpolarizablity}{6.2353808e-65}{\raiseto{4}\coulomb\raiseto{4}\metre\per\cube\joule}{1.1e-71}{6.2353808(11)e-65}
\physicalconstants@declare{atomic unit of electric dipole moment}{8.47835309e-30}{\coulomb\metre}{7.3e-37}{8.47835309(73)e-30}
\physicalconstants@declare{atomic unit of electric polarizablity}{1.648777274e-41}{\square\coulomb\square\metre\per\joule}{1.6e-49}{1.648777274(16)e-41}
\physicalconstants@declare{atomic unit of electric quadrupole moment}{4.48655124e-40}{\coulomb\square\metre}{3.9e-47}{4.48655124(39)e-40}
\physicalconstants@declare{atomic unit of magn. dipole moment}{1.8548019e-23}{\joule\per\tesla}{1.6e-30}{1.8548019(2)e-23}
\physicalconstants@declare{atomic unit of magn. flux density}{235051.756758}{\tesla}{7.1e-05}{235051.756758(71)}
\physicalconstants@declare{deuteron magn. moment}{4.33073482e-27}{\joule\per\tesla}{3.8e-34}{4.33073482(38)e-27}
\physicalconstants@declare{deuteron magn. moment to Bohr magneton ratio}{0.0004669754567}{}{5e-12}{0.0004669754567(50)}
\physicalconstants@declare{deuteron magn. moment to nuclear magneton ratio}{0.8574382329}{}{9.2e-09}{0.8574382329(92)}
\physicalconstants@declare{deuteron-electron magn. moment ratio}{-0.0004664345548}{}{5e-12}{-0.0004664345548(50)}
\physicalconstants@declare{deuteron-proton magn. moment ratio}{0.3070122084}{}{4.5e-09}{0.3070122084(45)}
\physicalconstants@declare{deuteron-neutron magn. moment ratio}{-0.44820652}{}{1.1e-07}{-0.44820652(11)}
\physicalconstants@declare{electron gyromagn. ratio}{176085963023.0}{\per\second\per\tesla}{53.0}{176085963023.0(530)}
\physicalconstants@declare{electron gyromagn. ratio over 2 pi}{28024.9532}{\mega\hertz\per\tesla}{0.0024}{28024.9532(00024)}
\physicalconstants@declare{electron magn. moment}{-9.28476412e-24}{\joule\per\tesla}{8e-31}{-9.28476412(80)e-24}
\physicalconstants@declare{electron magn. moment to Bohr magneton ratio}{-1.0011596521859}{}{3.8e-12}{-1.0011596521859(38)}
\physicalconstants@declare{electron magn. moment to nuclear magneton ratio}{-1838.28197107}{}{8.5e-07}{-1838.28197107(85)}
\physicalconstants@declare{electron magn. moment anomaly}{0.0011596521859}{}{3.8e-12}{0.0011596521859(38)}
\physicalconstants@declare{electron to shielded proton magn. moment ratio}{-658.2275956}{}{7.1e-06}{-658.2275956(71)}
\physicalconstants@declare{electron to shielded helion magn. moment ratio}{864.058255}{}{1e-05}{864.058255(10)}
\physicalconstants@declare{electron-deuteron magn. moment ratio}{-2143.923493}{}{2.3e-05}{-2143.923493(23)}
\physicalconstants@declare{electron-muon magn. moment ratio}{206.7669894}{}{5.4e-06}{206.7669894(54)}
\physicalconstants@declare{electron-neutron magn. moment ratio}{960.9205}{}{0.00023}{960.9205(3)}
\physicalconstants@declare{electron-proton magn. moment ratio}{-658.2106862}{}{6.6e-06}{-658.2106862(66)}
\physicalconstants@declare{magn. constant}{1.2566370614e-06}{\newton\per\square\ampere}{0.0}{1.2566370614(0)e-06}
\physicalconstants@declare{magn. flux quantum}{2.067833848e-15}{\weber}{0.0}{2.067833848(0)e-15}
\physicalconstants@declare{muon magn. moment}{-4.49044799e-26}{\joule\per\tesla}{4e-33}{-4.49044799(40)e-26}
\physicalconstants@declare{muon magn. moment to Bohr magneton ratio}{-0.00484197045}{}{1.3e-10}{-0.00484197045(13)}
\physicalconstants@declare{muon magn. moment to nuclear magneton ratio}{-8.89059698}{}{2.3e-07}{-8.89059698(23)}
\physicalconstants@declare{muon-proton magn. moment ratio}{-3.183345118}{}{8.9e-08}{-3.183345118(89)}
\physicalconstants@declare{neutron gyromagn. ratio}{183247171.0}{\per\second\per\tesla}{43.0}{183247171.0(430)}
\physicalconstants@declare{neutron gyromagn. ratio over 2 pi}{29.164695}{\mega\hertz\per\tesla}{7.3e-06}{29.164695(8)}
\physicalconstants@declare{neutron magn. moment}{-9.6623645e-27}{\joule\per\tesla}{2.4e-33}{-9.6623645(24)e-27}
\physicalconstants@declare{neutron magn. moment to Bohr magneton ratio}{-0.00104187563}{}{2.5e-10}{-0.00104187563(25)}
\physicalconstants@declare{neutron magn. moment to nuclear magneton ratio}{-1.91304273}{}{4.5e-07}{-1.91304273(45)}
\physicalconstants@declare{neutron to shielded proton magn. moment ratio}{-0.68499694}{}{1.6e-07}{-0.68499694(16)}
\physicalconstants@declare{neutron-electron magn. moment ratio}{0.00104066882}{}{2.5e-10}{0.00104066882(25)}
\physicalconstants@declare{neutron-proton magn. moment ratio}{-0.68497934}{}{1.6e-07}{-0.68497934(16)}
\physicalconstants@declare{proton gyromagn. ratio}{267522187.44}{\per\second\per\tesla}{0.11}{267522187.44(011)}
\physicalconstants@declare{proton gyromagn. ratio over 2 pi}{42.5774813}{\mega\hertz\per\tesla}{3.7e-06}{42.5774813(37)}
\physicalconstants@declare{proton magn. moment}{1.41060671e-26}{\joule\per\tesla}{1.2e-33}{1.41060671(12)e-26}
\physicalconstants@declare{proton magn. moment to Bohr magneton ratio}{0.001521032206}{}{1.5e-11}{0.001521032206(15)}
\physicalconstants@declare{proton magn. moment to nuclear magneton ratio}{2.792847351}{}{2.8e-08}{2.792847351(28)}
\physicalconstants@declare{proton magn. shielding correction}{2.5689e-05}{}{1.1e-08}{2.5689(11)e-05}
\physicalconstants@declare{proton-neutron magn. moment ratio}{-1.45989805}{}{3.4e-07}{-1.45989805(34)}
\physicalconstants@declare{shielded helion gyromagn. ratio}{203789456.9}{\per\second\per\tesla}{2.4}{203789456.9(24)}
\physicalconstants@declare{shielded helion gyromagn. ratio over 2 pi}{32.4341015}{\mega\hertz\per\tesla}{2.8e-06}{32.4341015(28)}
\physicalconstants@declare{shielded helion magn. moment}{-1.074553024e-26}{\joule\per\tesla}{9.3e-34}{-1.074553024(93)e-26}
\physicalconstants@declare{shielded helion magn. moment to Bohr magneton ratio}{-0.001158671474}{}{1.4e-11}{-0.001158671474(14)}
\physicalconstants@declare{shielded helion magn. moment to nuclear magneton ratio}{-2.127497723}{}{2.5e-08}{-2.127497723(25)}
\physicalconstants@declare{shielded helion to proton magn. moment ratio}{-0.761766562}{}{1.2e-08}{-0.761766562(12)}
\physicalconstants@declare{shielded helion to shielded proton magn. moment ratio}{-0.7617861313}{}{3.3e-09}{-0.7617861313(33)}
\physicalconstants@declare{shielded proton magn. moment}{1.41057047e-26}{\joule\per\tesla}{1.2e-33}{1.41057047(12)e-26}
\physicalconstants@declare{shielded proton magn. moment to Bohr magneton ratio}{0.001520993132}{}{1.6e-11}{0.001520993132(16)}
\physicalconstants@declare{shielded proton magn. moment to nuclear magneton ratio}{2.792775604}{}{3e-08}{2.792775604(30)}
\physicalconstants@declare{{220} lattice spacing of silicon}{1.920155714e-10}{\metre}{3.2e-18}{1.920155714(32)e-10}
\physicalconstants@declare{lattice spacing of silicon}{1.920155762e-10}{\metre}{5e-18}{1.920155762(50)e-10}
\physicalconstants@declare{alpha particle-electron mass ratio}{7294.29954142}{}{2.4e-07}{7294.29954142(24)}
\physicalconstants@declare{alpha particle mass}{6.6446573357e-27}{\kilogram}{2e-36}{6.6446573357(20)e-27}
\physicalconstants@declare{alpha particle mass energy equivalent}{5.9719201914e-10}{\joule}{1.8e-19}{5.9719201914(18)e-10}
\physicalconstants@declare{alpha particle mass energy equivalent in MeV}{3727.3794066}{\mega\electronvolt}{1.1e-06}{3727.3794066(11)}
\physicalconstants@declare{alpha particle mass in u}{4.001506179127}{\atomicmassunit}{6.3e-11}{4.001506179127(63)}
\physicalconstants@declare{alpha particle molar mass}{0.0040015061777}{\kilogram\per\mole}{1.2e-12}{0.0040015061777(12)}
\physicalconstants@declare{alpha particle-proton mass ratio}{3.97259969009}{}{2.2e-10}{3.97259969009(22)}
\physicalconstants@declare{Angstrom star}{1.00001495e-10}{\metre}{9e-17}{1.00001495(90)e-10}
\physicalconstants@declare{atomic mass constant}{1.6605390666e-27}{\kilogram}{5e-37}{1.6605390666(5)e-27}
\physicalconstants@declare{atomic mass constant energy equivalent}{1.4924180856e-10}{\joule}{4.5e-20}{1.4924180856(5)e-10}
\physicalconstants@declare{atomic mass constant energy equivalent in MeV}{931.49410242}{\mega\electronvolt}{2.8e-07}{931.49410242(28)}
\physicalconstants@declare{atomic mass unit-electron volt relationship}{931494102.42}{\electronvolt}{0.28}{931494102.42(028)}
\physicalconstants@declare{atomic mass unit-hartree relationship}{34231776.874}{}{0.01}{34231776.874(0010)}
\physicalconstants@declare{atomic mass unit-hertz relationship}{2.25234271871e+23}{\hertz}{68000000000000.0}{2.25234271871(68)e+23}
\physicalconstants@declare{atomic mass unit-inverse meter relationship}{751300661040000.0}{\per\metre}{230000.0}{751300661040000.0(2300000)}
\physicalconstants@declare{atomic mass unit-joule relationship}{1.4924180856e-10}{\joule}{4.5e-20}{1.4924180856(5)e-10}
\physicalconstants@declare{atomic mass unit-kelvin relationship}{10809540191600.0}{\kelvin}{3300.0}{10809540191600.0(33000)}
\physicalconstants@declare{atomic mass unit-kilogram relationship}{1.6605390666e-27}{\kilogram}{5e-37}{1.6605390666(5)e-27}
\physicalconstants@declare{atomic unit of 1st hyperpolarizability}{3.2063613061e-53}{\cube\coulomb\cube\metre\per\square\joule}{1.5e-62}{3.2063613061(15)e-53}
\physicalconstants@declare{atomic unit of 2nd hyperpolarizability}{6.2353799905e-65}{\raiseto{4}\coulomb\raiseto{4}\metre\per\cube\joule}{3.8e-74}{6.2353799905(38)e-65}
\physicalconstants@declare{atomic unit of action}{1.054571817e-34}{\joule\second}{0.0}{1.054571817(0)e-34}
\physicalconstants@declare{atomic unit of charge}{1.602176634e-19}{\coulomb}{0.0}{1.602176634(0)e-19}
\physicalconstants@declare{atomic unit of charge density}{1081202384570.0}{\coulomb\per\cube\metre}{490.0}{1081202384570.0(4900)}
\physicalconstants@declare{atomic unit of current}{0.00662361823751}{\ampere}{1.3e-14}{0.00662361823751(2)}
\physicalconstants@declare{atomic unit of electric dipole mom.}{8.4783536255e-30}{\coulomb\metre}{1.3e-39}{8.4783536255(13)e-30}
\physicalconstants@declare{atomic unit of electric field}{514220674763.0}{\volt\per\metre}{78.0}{514220674763.0(780)}
\physicalconstants@declare{atomic unit of electric field gradient}{9.7173624292e+21}{\volt\per\square\metre}{2900000000000.0}{9.7173624292(29)e+21}
\physicalconstants@declare{atomic unit of electric polarizability}{1.64877727436e-41}{\square\coulomb\square\metre\per\joule}{5e-51}{1.64877727436(50)e-41}
\physicalconstants@declare{atomic unit of electric potential}{27.211386245988}{\volt}{5.3e-11}{27.211386245988(53)}
\physicalconstants@declare{atomic unit of electric quadrupole mom.}{4.4865515246e-40}{\coulomb\square\metre}{1.4e-49}{4.4865515246(14)e-40}
\physicalconstants@declare{atomic unit of energy}{4.3597447222071e-18}{\joule}{8.5e-30}{4.3597447222071(85)e-18}
\physicalconstants@declare{atomic unit of force}{8.2387234983e-08}{\newton}{1.2e-17}{8.2387234983(12)e-08}
\physicalconstants@declare{atomic unit of length}{5.29177210903e-11}{\metre}{8e-21}{5.29177210903(80)e-11}
\physicalconstants@declare{atomic unit of mag. dipole mom.}{1.85480201566e-23}{\joule\per\tesla}{5.6e-33}{1.85480201566(56)e-23}
\physicalconstants@declare{atomic unit of mag. flux density}{235051.756758}{\tesla}{7.1e-05}{235051.756758(71)}
\physicalconstants@declare{atomic unit of magnetizability}{7.8910366008e-29}{\joule\per\square\tesla}{4.8e-38}{7.8910366008(48)e-29}
\physicalconstants@declare{atomic unit of mass}{9.1093837015e-31}{\kilogram}{2.8e-40}{9.1093837015(28)e-31}
\physicalconstants@declare{atomic unit of momentum}{1.9928519141e-24}{\kilogram\metre\per\second}{3e-34}{1.9928519141(3)e-24}
\physicalconstants@declare{atomic unit of permittivity}{1.11265005545e-10}{\farad\per\metre}{1.7e-20}{1.11265005545(17)e-10}
\physicalconstants@declare{atomic unit of time}{2.4188843265857e-17}{\second}{4.7e-29}{2.4188843265857(47)e-17}
\physicalconstants@declare{atomic unit of velocity}{2187691.26364}{\metre\per\second}{0.00033}{2187691.26364(000033)}
\physicalconstants@declare{Avogadro constant}{6.02214076e+23}{\per\mole}{0.0}{6.02214076(0)e+23}
\physicalconstants@declare{Bohr magneton}{9.2740100783e-24}{\joule\per\tesla}{2.8e-33}{9.2740100783(28)e-24}
\physicalconstants@declare{Bohr magneton in eV/T}{5.788381806e-05}{\electronvolt\per\tesla}{1.7e-14}{5.788381806(2)e-05}
\physicalconstants@declare{Bohr magneton in Hz/T}{13996244936.1}{\hertz\per\tesla}{4.2}{13996244936.1(42)}
\physicalconstants@declare{Bohr magneton in inverse meters per tesla}{46.68644814}{\per\metre\per\tesla}{2.9e-07}{46.68644814(29)}
\physicalconstants@declare{Bohr magneton in K/T}{0.67171381563}{\kelvin\per\tesla}{2e-10}{0.67171381563(20)}
\physicalconstants@declare{Bohr radius}{5.29177210903e-11}{\metre}{8e-21}{5.29177210903(80)e-11}
\physicalconstants@declare{Boltzmann constant}{1.380649e-23}{\joule\per\kelvin}{0.0}{1.380649(0)e-23}
\physicalconstants@declare{Boltzmann constant in eV/K}{8.617333262e-05}{\electronvolt\per\kelvin}{0.0}{8.617333262(0)e-05}
\physicalconstants@declare{Boltzmann constant in Hz/K}{20836619120.0}{\hertz\per\kelvin}{0.0}{20836619120.0(0)}
\physicalconstants@declare{Boltzmann constant in inverse meters per kelvin}{69.503457}{\per\metre\per\kelvin}{4e-05}{69.503457(40)}
\physicalconstants@declare{characteristic impedance of vacuum}{376.73031366686166}{\ohm}{5.61366546036269e-08}{376.73031366686166(5613666)}
\physicalconstants@declare{classical electron radius}{2.8179403262e-15}{\metre}{1.3e-24}{2.8179403262(13)e-15}
\physicalconstants@declare{Compton wavelength}{2.42631023867e-12}{\metre}{7.3e-22}{2.42631023867(73)e-12}
\physicalconstants@declare{Compton wavelength over 2 pi}{3.8615926764e-13}{\metre}{1.8e-22}{3.8615926764(18)e-13}
\physicalconstants@declare{conductance quantum}{7.748091729e-05}{\siemens}{0.0}{7.748091729(0)e-05}
\physicalconstants@declare{conventional value of Josephson constant}{483597900000000.0}{\hertz\per\volt}{0.0}{483597900000000.0(0)}
\physicalconstants@declare{conventional value of von Klitzing constant}{25812.807}{\ohm}{0.0}{25812.807(0)}
\physicalconstants@declare{Cu x unit}{1.00207697e-13}{\metre}{2.8e-20}{1.00207697(28)e-13}
\physicalconstants@declare{deuteron-electron mag. mom. ratio}{-0.0004664345551}{}{1.2e-12}{-0.0004664345551(12)}
\physicalconstants@declare{deuteron-electron mass ratio}{3670.48296788}{}{1.3e-07}{3670.48296788(13)}
\physicalconstants@declare{deuteron g factor}{0.8574382338}{}{2.2e-09}{0.8574382338(22)}
\physicalconstants@declare{deuteron mag. mom.}{4.330735094e-27}{\joule\per\tesla}{1.1e-35}{4.330735094(11)e-27}
\physicalconstants@declare{deuteron mag. mom. to Bohr magneton ratio}{0.000466975457}{}{1.2e-12}{0.000466975457(2)}
\physicalconstants@declare{deuteron mag. mom. to nuclear magneton ratio}{0.8574382338}{}{2.2e-09}{0.8574382338(22)}
\physicalconstants@declare{deuteron mass}{3.3435837724e-27}{\kilogram}{1e-36}{3.3435837724(10)e-27}
\physicalconstants@declare{deuteron mass energy equivalent}{3.00506323102e-10}{\joule}{9.1e-20}{3.00506323102(91)e-10}
\physicalconstants@declare{deuteron mass energy equivalent in MeV}{1875.61294257}{\mega\electronvolt}{5.7e-07}{1875.61294257(57)}
\physicalconstants@declare{deuteron mass in u}{2.013553212745}{\atomicmassunit}{4e-11}{2.013553212745(40)}
\physicalconstants@declare{deuteron molar mass}{0.00201355321205}{\kilogram\per\mole}{6.1e-13}{0.00201355321205(61)}
\physicalconstants@declare{deuteron-neutron mag. mom. ratio}{-0.44820653}{}{1.1e-07}{-0.44820653(11)}
\physicalconstants@declare{deuteron-proton mag. mom. ratio}{0.30701220939}{}{7.9e-10}{0.30701220939(79)}
\physicalconstants@declare{deuteron-proton mass ratio}{1.99900750139}{}{1.1e-10}{1.99900750139(11)}
\physicalconstants@declare{deuteron rms charge radius}{2.12799e-15}{\metre}{7.4e-19}{2.12799(74)e-15}
\physicalconstants@declare{electric constant}{8.8541878128e-12}{\farad\per\metre}{1.3e-21}{8.8541878128(13)e-12}
\physicalconstants@declare{electron charge to mass quotient}{-175882001076.0}{\coulomb\per\kilogram}{53.0}{-175882001076.0(530)}
\physicalconstants@declare{electron-deuteron mag. mom. ratio}{-2143.9234915}{}{5.6e-06}{-2143.9234915(56)}
\physicalconstants@declare{electron-deuteron mass ratio}{0.0002724437107462}{}{9.6e-15}{0.0002724437107462(96)}
\physicalconstants@declare{electron g factor}{-2.00231930436256}{}{3.5e-13}{-2.00231930436256(35)}
\physicalconstants@declare{electron gyromag. ratio}{176085963023.0}{\per\second\per\tesla}{53.0}{176085963023.0(530)}
\physicalconstants@declare{electron gyromag. ratio over 2 pi}{28024.95164}{\mega\hertz\per\tesla}{0.00017}{28024.95164(000017)}
\physicalconstants@declare{electron mag. mom.}{-9.2847647043e-24}{\joule\per\tesla}{2.8e-33}{-9.2847647043(28)e-24}
\physicalconstants@declare{electron mag. mom. anomaly}{0.00115965218128}{}{1.8e-13}{0.00115965218128(18)}
\physicalconstants@declare{electron mag. mom. to Bohr magneton ratio}{-1.00115965218128}{}{1.8e-13}{-1.00115965218128(18)}
\physicalconstants@declare{electron mag. mom. to nuclear magneton ratio}{-1838.28197188}{}{1.1e-07}{-1838.28197188(11)}
\physicalconstants@declare{electron mass}{9.1093837015e-31}{\kilogram}{2.8e-40}{9.1093837015(28)e-31}
\physicalconstants@declare{electron mass energy equivalent}{8.1871057769e-14}{\joule}{2.5e-23}{8.1871057769(25)e-14}
\physicalconstants@declare{electron mass energy equivalent in MeV}{0.51099895}{\mega\electronvolt}{1.5e-10}{0.51099895(1)}
\physicalconstants@declare{electron mass in u}{0.000548579909065}{\atomicmassunit}{1.6e-14}{0.000548579909065(16)}
\physicalconstants@declare{electron molar mass}{5.4857990888e-07}{\kilogram\per\mole}{1.7e-16}{5.4857990888(17)e-07}
\physicalconstants@declare{electron-muon mag. mom. ratio}{206.7669883}{}{4.6e-06}{206.7669883(46)}
\physicalconstants@declare{electron-muon mass ratio}{0.00483633169}{}{1.1e-10}{0.00483633169(11)}
\physicalconstants@declare{electron-neutron mag. mom. ratio}{960.9205}{}{0.00023}{960.9205(3)}
\physicalconstants@declare{electron-neutron mass ratio}{0.00054386734424}{}{2.6e-13}{0.00054386734424(26)}
\physicalconstants@declare{electron-proton mag. mom. ratio}{-658.21068789}{}{2e-07}{-658.21068789(20)}
\physicalconstants@declare{electron-proton mass ratio}{0.000544617021487}{}{3.3e-14}{0.000544617021487(33)}
\physicalconstants@declare{electron-tau mass ratio}{0.000287585}{}{1.9e-08}{0.000287585(19)}
\physicalconstants@declare{electron to alpha particle mass ratio}{0.0001370933554787}{}{4.5e-15}{0.0001370933554787(45)}
\physicalconstants@declare{electron to shielded helion mag. mom. ratio}{864.058257}{}{1e-05}{864.058257(10)}
\physicalconstants@declare{electron to shielded proton mag. mom. ratio}{-658.2275971}{}{7.2e-06}{-658.2275971(72)}
\physicalconstants@declare{electron volt}{1.602176634e-19}{\joule}{0.0}{1.602176634(0)e-19}
\physicalconstants@declare{electron volt-atomic mass unit relationship}{1.07354410233e-09}{\atomicmassunit}{3.2e-19}{1.07354410233(32)e-09}
\physicalconstants@declare{electron volt-hartree relationship}{0.036749322175655}{}{7.1e-14}{0.036749322175655(71)}
\physicalconstants@declare{electron volt-hertz relationship}{241798924200000.0}{\hertz}{0.0}{241798924200000.0(0)}
\physicalconstants@declare{electron volt-inverse meter relationship}{806554.3937}{\per\metre}{0.0}{806554.3937(0)}
\physicalconstants@declare{electron volt-joule relationship}{1.602176634e-19}{\joule}{0.0}{1.602176634(0)e-19}
\physicalconstants@declare{electron volt-kelvin relationship}{11604.51812}{\kelvin}{0.0}{11604.51812(0)}
\physicalconstants@declare{electron volt-kilogram relationship}{1.782661921e-36}{\kilogram}{0.0}{1.782661921(0)e-36}
\physicalconstants@declare{elementary charge}{1.602176634e-19}{\coulomb}{0.0}{1.602176634(0)e-19}
\physicalconstants@declare{elementary charge over h}{241798926200000.0}{\ampere\per\joule}{1500000.0}{241798926200000.0(15000000)}
\physicalconstants@declare{Faraday constant}{96485.33212}{\coulomb\per\mole}{0.0}{96485.33212(0)}
\physicalconstants@declare{Faraday constant for conventional electric current}{96485.3251}{\per\mole}{0.0012}{96485.3251(00012)}
\physicalconstants@declare{Fermi coupling constant}{1.1663787e-05}{\per\square\giga\electronvolt}{6e-12}{1.1663787(6)e-05}
\physicalconstants@declare{fine-structure constant}{0.0072973525693}{}{1.1e-12}{0.0072973525693(11)}
\physicalconstants@declare{first radiation constant}{3.741771852e-16}{\watt\square\metre}{0.0}{3.741771852(0)e-16}
\physicalconstants@declare{first radiation constant for spectral radiance}{1.191042972e-16}{\watt\square\metre\per\steradian}{0.0}{1.191042972(0)e-16}
\physicalconstants@declare{hartree-atomic mass unit relationship}{2.92126232205e-08}{\atomicmassunit}{8.8e-18}{2.92126232205(88)e-08}
\physicalconstants@declare{hartree-electron volt relationship}{27.211386245988}{\electronvolt}{5.3e-11}{27.211386245988(53)}
\physicalconstants@declare{Hartree energy}{4.3597447222071e-18}{\joule}{8.5e-30}{4.3597447222071(85)e-18}
\physicalconstants@declare{Hartree energy in eV}{27.211386245988}{\electronvolt}{5.3e-11}{27.211386245988(53)}
\physicalconstants@declare{hartree-hertz relationship}{6579683920502000.0}{\hertz}{13000.0}{6579683920502000.0(130000)}
\physicalconstants@declare{hartree-inverse meter relationship}{21947463.13632}{\per\metre}{4.3e-05}{21947463.13632(5)}
\physicalconstants@declare{hartree-joule relationship}{4.3597447222071e-18}{\joule}{8.5e-30}{4.3597447222071(85)e-18}
\physicalconstants@declare{hartree-kelvin relationship}{315775.02480407}{\kelvin}{6.1e-07}{315775.02480407(61)}
\physicalconstants@declare{hartree-kilogram relationship}{4.8508702095432e-35}{\kilogram}{9.4e-47}{4.8508702095432(94)e-35}
\physicalconstants@declare{helion-electron mass ratio}{5495.88528007}{}{2.4e-07}{5495.88528007(24)}
\physicalconstants@declare{helion mass}{5.0064127796e-27}{\kilogram}{1.5e-36}{5.0064127796(15)e-27}
\physicalconstants@declare{helion mass energy equivalent}{4.4995394125e-10}{\joule}{1.4e-19}{4.4995394125(14)e-10}
\physicalconstants@declare{helion mass energy equivalent in MeV}{2808.39160743}{\mega\electronvolt}{8.5e-07}{2808.39160743(85)}
\physicalconstants@declare{helion mass in u}{3.014932247175}{\atomicmassunit}{9.7e-11}{3.014932247175(97)}
\physicalconstants@declare{helion molar mass}{0.00301493224613}{\kilogram\per\mole}{9.1e-13}{0.00301493224613(91)}
\physicalconstants@declare{helion-proton mass ratio}{2.99315267167}{}{1.3e-10}{2.99315267167(13)}
\physicalconstants@declare{hertz-atomic mass unit relationship}{4.4398216652e-24}{\atomicmassunit}{1.3e-33}{4.4398216652(13)e-24}
\physicalconstants@declare{hertz-electron volt relationship}{4.135667696e-15}{\electronvolt}{0.0}{4.135667696(0)e-15}
\physicalconstants@declare{hertz-hartree relationship}{1.519829846057e-16}{}{2.9e-28}{1.519829846057(3)e-16}
\physicalconstants@declare{hertz-inverse meter relationship}{3.3356409519815204e-09}{\per\metre}{0.0}{3.3356409519815204(0)e-09}
\physicalconstants@declare{hertz-joule relationship}{6.62607015e-34}{\joule}{0.0}{6.62607015(0)e-34}
\physicalconstants@declare{hertz-kelvin relationship}{4.799243073e-11}{\kelvin}{0.0}{4.799243073(0)e-11}
\physicalconstants@declare{hertz-kilogram relationship}{7.372497323e-51}{\kilogram}{0.0}{7.372497323(0)e-51}
\physicalconstants@declare{inverse fine-structure constant}{137.035999084}{}{2.1e-08}{137.035999084(21)}
\physicalconstants@declare{inverse meter-atomic mass unit relationship}{1.3310250501e-15}{\atomicmassunit}{4e-25}{1.3310250501(4)e-15}
\physicalconstants@declare{inverse meter-electron volt relationship}{1.239841984e-06}{\electronvolt}{0.0}{1.239841984(0)e-06}
\physicalconstants@declare{inverse meter-hartree relationship}{4.556335252912e-08}{}{8.8e-20}{4.556335252912(9)e-08}
\physicalconstants@declare{inverse meter-hertz relationship}{299792458.0}{\hertz}{0.0}{299792458.0(0)}
\physicalconstants@declare{inverse meter-joule relationship}{1.986445857e-25}{\joule}{0.0}{1.986445857(0)e-25}
\physicalconstants@declare{inverse meter-kelvin relationship}{0.01438776877}{\kelvin}{0.0}{0.01438776877(0)}
\physicalconstants@declare{inverse meter-kilogram relationship}{2.210219094e-42}{\kilogram}{0.0}{2.210219094(0)e-42}
\physicalconstants@declare{inverse of conductance quantum}{12906.40372}{\ohm}{0.0}{12906.40372(0)}
\physicalconstants@declare{Josephson constant}{483597848400000.0}{\hertz\per\volt}{0.0}{483597848400000.0(0)}
\physicalconstants@declare{joule-atomic mass unit relationship}{6700535256.5}{\atomicmassunit}{2.0}{6700535256.5(20)}
\physicalconstants@declare{joule-electron volt relationship}{6.241509074e+18}{\electronvolt}{0.0}{6.241509074(0)e+18}
\physicalconstants@declare{joule-hartree relationship}{2.2937122783963e+17}{}{450000.0}{2.2937122783963(45)e+17}
\physicalconstants@declare{joule-hertz relationship}{1.509190179e+33}{\hertz}{0.0}{1.509190179(0)e+33}
\physicalconstants@declare{joule-inverse meter relationship}{5.034116567e+24}{\per\metre}{0.0}{5.034116567(0)e+24}
\physicalconstants@declare{joule-kelvin relationship}{7.242970516e+22}{\kelvin}{0.0}{7.242970516(0)e+22}
\physicalconstants@declare{joule-kilogram relationship}{1.1126500560536185e-17}{\kilogram}{0.0}{1.1126500560536185(0)e-17}
\physicalconstants@declare{kelvin-atomic mass unit relationship}{9.2510873014e-14}{\atomicmassunit}{2.8e-23}{9.2510873014(28)e-14}
\physicalconstants@declare{kelvin-electron volt relationship}{8.617333262e-05}{\electronvolt}{0.0}{8.617333262(0)e-05}
\physicalconstants@declare{kelvin-hartree relationship}{3.1668115634556e-06}{}{6.1e-18}{3.1668115634556(61)e-06}
\physicalconstants@declare{kelvin-hertz relationship}{20836619120.0}{\hertz}{0.0}{20836619120.0(0)}
\physicalconstants@declare{kelvin-inverse meter relationship}{69.50348004}{\per\metre}{0.0}{69.50348004(0)}
\physicalconstants@declare{kelvin-joule relationship}{1.380649e-23}{\joule}{0.0}{1.380649(0)e-23}
\physicalconstants@declare{kelvin-kilogram relationship}{1.536179187e-40}{\kilogram}{0.0}{1.536179187(0)e-40}
\physicalconstants@declare{kilogram-atomic mass unit relationship}{6.0221407621e+26}{\atomicmassunit}{1.8e+17}{6.0221407621(18)e+26}
\physicalconstants@declare{kilogram-electron volt relationship}{5.609588603e+35}{\electronvolt}{0.0}{5.609588603(0)e+35}
\physicalconstants@declare{kilogram-hartree relationship}{2.0614857887409e+34}{}{4e+22}{2.0614857887409(40)e+34}
\physicalconstants@declare{kilogram-hertz relationship}{1.356392489e+50}{\hertz}{0.0}{1.356392489(0)e+50}
\physicalconstants@declare{kilogram-inverse meter relationship}{4.524438335e+41}{\per\metre}{0.0}{4.524438335(0)e+41}
\physicalconstants@declare{kilogram-joule relationship}{8.987551787368176e+16}{\joule}{0.0}{8.987551787368176(0)e+16}
\physicalconstants@declare{kilogram-kelvin relationship}{6.50965726e+39}{\kelvin}{0.0}{6.50965726(0)e+39}
\physicalconstants@declare{lattice parameter of silicon}{5.431020511e-10}{\metre}{8.9e-18}{5.431020511(89)e-10}
\physicalconstants@declare{Loschmidt constant (273.15 K, 101.325 kPa)}{2.686780111e+25}{\per\cube\metre}{0.0}{2.686780111(0)e+25}
\physicalconstants@declare{mag. constant}{1.25663706212e-06}{\newton\per\square\ampere}{1.9e-16}{1.25663706212(19)e-06}
\physicalconstants@declare{mag. flux quantum}{2.067833848e-15}{\weber}{0.0}{2.067833848(0)e-15}
\physicalconstants@declare{molar gas constant}{8.314462618}{\joule\per\mole\per\kelvin}{0.0}{8.314462618(0)}
\physicalconstants@declare{molar mass constant}{0.00099999999965}{\kilogram\per\mole}{3e-13}{0.00099999999965(30)}
\physicalconstants@declare{molar mass of carbon-12}{0.0119999999958}{\kilogram\per\mole}{3.6e-12}{0.0119999999958(36)}
\physicalconstants@declare{molar Planck constant}{3.990312712e-10}{\joule\per\hertz\per\mole}{0.0}{3.990312712(0)e-10}
\physicalconstants@declare{molar Planck constant times c}{0.119626565582}{\joule\metre\per\mole}{5.4e-11}{0.119626565582(54)}
\physicalconstants@declare{molar volume of ideal gas (273.15 K, 100 kPa)}{0.02271095464}{\cube\metre\per\mole}{0.0}{0.02271095464(0)}
\physicalconstants@declare{molar volume of ideal gas (273.15 K, 101.325 kPa)}{0.02241396954}{\cube\metre\per\mole}{0.0}{0.02241396954(0)}
\physicalconstants@declare{molar volume of silicon}{1.205883199e-05}{\cube\metre\per\mole}{6e-13}{1.205883199(60)e-05}
\physicalconstants@declare{Mo x unit}{1.00209952e-13}{\metre}{5.3e-20}{1.00209952(53)e-13}
\physicalconstants@declare{muon Compton wavelength}{1.17344411e-14}{\metre}{2.6e-22}{1.17344411(3)e-14}
\physicalconstants@declare{muon Compton wavelength over 2 pi}{1.867594308e-15}{\metre}{4.2e-23}{1.867594308(42)e-15}
\physicalconstants@declare{muon-electron mass ratio}{206.768283}{}{4.6e-06}{206.768283(5)}
\physicalconstants@declare{muon g factor}{-2.0023318418}{}{1.3e-09}{-2.0023318418(13)}
\physicalconstants@declare{muon mag. mom.}{-4.4904483e-26}{\joule\per\tesla}{1e-33}{-4.4904483(1)e-26}
\physicalconstants@declare{muon mag. mom. anomaly}{0.00116592089}{}{6.3e-10}{0.00116592089(63)}
\physicalconstants@declare{muon mag. mom. to Bohr magneton ratio}{-0.00484197047}{}{1.1e-10}{-0.00484197047(11)}
\physicalconstants@declare{muon mag. mom. to nuclear magneton ratio}{-8.89059703}{}{2e-07}{-8.89059703(20)}
\physicalconstants@declare{muon mass}{1.883531627e-28}{\kilogram}{4.2e-36}{1.883531627(42)e-28}
\physicalconstants@declare{muon mass energy equivalent}{1.692833804e-11}{\joule}{3.8e-19}{1.692833804(38)e-11}
\physicalconstants@declare{muon mass energy equivalent in MeV}{105.6583755}{\mega\electronvolt}{2.3e-06}{105.6583755(23)}
\physicalconstants@declare{muon mass in u}{0.1134289259}{\atomicmassunit}{2.5e-09}{0.1134289259(25)}
\physicalconstants@declare{muon molar mass}{0.0001134289259}{\kilogram\per\mole}{2.5e-12}{0.0001134289259(25)}
\physicalconstants@declare{muon-neutron mass ratio}{0.112454517}{}{2.5e-09}{0.112454517(3)}
\physicalconstants@declare{muon-proton mag. mom. ratio}{-3.183345142}{}{7.1e-08}{-3.183345142(71)}
\physicalconstants@declare{muon-proton mass ratio}{0.1126095264}{}{2.5e-09}{0.1126095264(25)}
\physicalconstants@declare{muon-tau mass ratio}{0.0594635}{}{4e-06}{0.0594635(40)}
\physicalconstants@declare{natural unit of action}{1.054571817e-34}{\joule\second}{0.0}{1.054571817(0)e-34}
\physicalconstants@declare{natural unit of action in eV s}{6.582119569e-16}{\electronvolt\second}{0.0}{6.582119569(0)e-16}
\physicalconstants@declare{natural unit of energy}{8.1871057769e-14}{\joule}{2.5e-23}{8.1871057769(25)e-14}
\physicalconstants@declare{natural unit of energy in MeV}{0.51099895}{\mega\electronvolt}{1.5e-10}{0.51099895(1)}
\physicalconstants@declare{natural unit of length}{3.8615926796e-13}{\metre}{1.2e-22}{3.8615926796(12)e-13}
\physicalconstants@declare{natural unit of mass}{9.1093837015e-31}{\kilogram}{2.8e-40}{9.1093837015(28)e-31}
\physicalconstants@declare{natural unit of momentum}{2.730924488e-22}{\kilogram\metre\per\second}{3.4e-30}{2.730924488(34)e-22}
\physicalconstants@declare{natural unit of momentum in MeV/c}{0.5109989461}{\mega\electronvolt\clight}{3.1e-09}{0.5109989461(31)}
\physicalconstants@declare{natural unit of time}{1.28808866819e-21}{\second}{3.9e-31}{1.28808866819(39)e-21}
\physicalconstants@declare{natural unit of velocity}{299792458.0}{\metre\per\second}{0.0}{299792458.0(0)}
\physicalconstants@declare{neutron Compton wavelength}{1.31959090581e-15}{\metre}{7.5e-25}{1.31959090581(75)e-15}
\physicalconstants@declare{neutron Compton wavelength over 2 pi}{2.1001941536e-16}{\metre}{1.4e-25}{2.1001941536(14)e-16}
\physicalconstants@declare{neutron-electron mag. mom. ratio}{0.00104066882}{}{2.5e-10}{0.00104066882(25)}
\physicalconstants@declare{neutron-electron mass ratio}{1838.68366173}{}{8.9e-07}{1838.68366173(89)}
\physicalconstants@declare{neutron g factor}{-3.82608545}{}{9e-07}{-3.82608545(90)}
\physicalconstants@declare{neutron gyromag. ratio}{183247171.0}{\per\second\per\tesla}{43.0}{183247171.0(430)}
\physicalconstants@declare{neutron gyromag. ratio over 2 pi}{29.1646933}{\mega\hertz\per\tesla}{6.9e-06}{29.1646933(69)}
\physicalconstants@declare{neutron mag. mom.}{-9.6623651e-27}{\joule\per\tesla}{2.3e-33}{-9.6623651(23)e-27}
\physicalconstants@declare{neutron mag. mom. to Bohr magneton ratio}{-0.00104187563}{}{2.5e-10}{-0.00104187563(25)}
\physicalconstants@declare{neutron mag. mom. to nuclear magneton ratio}{-1.91304273}{}{4.5e-07}{-1.91304273(45)}
\physicalconstants@declare{neutron mass}{1.67492749804e-27}{\kilogram}{9.5e-37}{1.67492749804(95)e-27}
\physicalconstants@declare{neutron mass energy equivalent}{1.50534976287e-10}{\joule}{8.6e-20}{1.50534976287(86)e-10}
\physicalconstants@declare{neutron mass energy equivalent in MeV}{939.56542052}{\mega\electronvolt}{5.4e-07}{939.56542052(54)}
\physicalconstants@declare{neutron mass in u}{1.00866491595}{\atomicmassunit}{4.9e-10}{1.00866491595(49)}
\physicalconstants@declare{neutron molar mass}{0.0010086649156}{\kilogram\per\mole}{5.7e-13}{0.0010086649156(6)}
\physicalconstants@declare{neutron-muon mass ratio}{8.89248406}{}{2e-07}{8.89248406(20)}
\physicalconstants@declare{neutron-proton mag. mom. ratio}{-0.68497934}{}{1.6e-07}{-0.68497934(16)}
\physicalconstants@declare{neutron-proton mass ratio}{1.00137841931}{}{4.9e-10}{1.00137841931(49)}
\physicalconstants@declare{neutron-tau mass ratio}{0.528779}{}{3.6e-05}{0.528779(36)}
\physicalconstants@declare{neutron to shielded proton mag. mom. ratio}{-0.68499694}{}{1.6e-07}{-0.68499694(16)}
\physicalconstants@declare{Newtonian constant of gravitation}{6.6743e-11}{\cube\metre\per\kilogram\per\square\second}{1.5e-15}{6.6743(2)e-11}
\physicalconstants@declare{Newtonian constant of gravitation over h-bar c}{6.70883e-39}{\giga\electronvolt\square\clight}{1.5e-43}{6.70883(15)e-39}
\physicalconstants@declare{nuclear magneton}{5.0507837461e-27}{\joule\per\tesla}{1.5e-36}{5.0507837461(15)e-27}
\physicalconstants@declare{nuclear magneton in eV/T}{3.15245125844e-08}{\electronvolt\per\tesla}{9.6e-18}{3.15245125844(96)e-08}
\physicalconstants@declare{nuclear magneton in inverse meters per tesla}{0.02542623432}{\per\metre\per\tesla}{1.6e-10}{0.02542623432(16)}
\physicalconstants@declare{nuclear magneton in K/T}{0.00036582677756}{\kelvin\per\tesla}{1.1e-13}{0.00036582677756(11)}
\physicalconstants@declare{nuclear magneton in MHz/T}{7.6225932291}{\mega\hertz\per\tesla}{2.3e-09}{7.6225932291(23)}
\physicalconstants@declare{Planck constant}{6.62607015e-34}{\joule\per\hertz}{0.0}{6.62607015(0)e-34}
\physicalconstants@declare{Planck constant in eV s}{4.135667662e-15}{\electronvolt\second}{2.5e-23}{4.135667662(25)e-15}
\physicalconstants@declare{Planck constant over 2 pi}{1.0545718e-34}{\joule\second}{1.3e-42}{1.0545718(1)e-34}
\physicalconstants@declare{Planck constant over 2 pi in eV s}{6.582119514e-16}{\electronvolt\second}{4e-24}{6.582119514(40)e-16}
\physicalconstants@declare{Planck constant over 2 pi times c in MeV fm}{197.3269788}{\mega\electronvolt\femto\meter}{1.2e-06}{197.3269788(12)}
\physicalconstants@declare{Planck length}{1.616255e-35}{\metre}{1.8e-40}{1.616255(18)e-35}
\physicalconstants@declare{Planck mass}{2.176434e-08}{\kilogram}{2.4e-13}{2.176434(24)e-08}
\physicalconstants@declare{Planck mass energy equivalent in GeV}{1.22089e+19}{\giga\electronvolt}{140000000000000.0}{1.22089(2)e+19}
\physicalconstants@declare{Planck temperature}{1.416784e+32}{\kelvin}{1.6e+27}{1.416784(16)e+32}
\physicalconstants@declare{Planck time}{5.391247e-44}{\second}{6e-49}{5.391247(60)e-44}
\physicalconstants@declare{proton charge to mass quotient}{95788331.56}{\coulomb\per\kilogram}{0.029}{95788331.56(3)}
\physicalconstants@declare{proton Compton wavelength}{1.32140985539e-15}{\metre}{4e-25}{1.32140985539(40)e-15}
\physicalconstants@declare{proton Compton wavelength over 2 pi}{2.10308910109e-16}{\metre}{9.7e-26}{2.10308910109(97)e-16}
\physicalconstants@declare{proton-electron mass ratio}{1836.15267343}{}{1.1e-07}{1836.15267343(11)}
\physicalconstants@declare{proton g factor}{5.5856946893}{}{1.6e-09}{5.5856946893(16)}
\physicalconstants@declare{proton gyromag. ratio}{267522187.44}{\per\second\per\tesla}{0.11}{267522187.44(011)}
\physicalconstants@declare{proton gyromag. ratio over 2 pi}{42.57747892}{\mega\hertz\per\tesla}{2.9e-07}{42.57747892(29)}
\physicalconstants@declare{proton mag. mom.}{1.41060679736e-26}{\joule\per\tesla}{6e-36}{1.41060679736(60)e-26}
\physicalconstants@declare{proton mag. mom. to Bohr magneton ratio}{0.0015210322023}{}{4.6e-13}{0.0015210322023(5)}
\physicalconstants@declare{proton mag. mom. to nuclear magneton ratio}{2.79284734463}{}{8.2e-10}{2.79284734463(82)}
\physicalconstants@declare{proton mag. shielding correction}{2.5689e-05}{}{1.1e-08}{2.5689(11)e-05}
\physicalconstants@declare{proton mass}{1.67262192369e-27}{\kilogram}{5.1e-37}{1.67262192369(51)e-27}
\physicalconstants@declare{proton mass energy equivalent}{1.50327761598e-10}{\joule}{4.6e-20}{1.50327761598(46)e-10}
\physicalconstants@declare{proton mass energy equivalent in MeV}{938.27208816}{\mega\electronvolt}{2.9e-07}{938.27208816(29)}
\physicalconstants@declare{proton mass in u}{1.007276466621}{\atomicmassunit}{5.3e-11}{1.007276466621(53)}
\physicalconstants@declare{proton molar mass}{0.00100727646627}{\kilogram\per\mole}{3.1e-13}{0.00100727646627(31)}
\physicalconstants@declare{proton-muon mass ratio}{8.88024337}{}{2e-07}{8.88024337(20)}
\physicalconstants@declare{proton-neutron mag. mom. ratio}{-1.45989805}{}{3.4e-07}{-1.45989805(34)}
\physicalconstants@declare{proton-neutron mass ratio}{0.99862347812}{}{4.9e-10}{0.99862347812(49)}
\physicalconstants@declare{proton rms charge radius}{8.414e-16}{\metre}{1.9e-18}{8.414(19)e-16}
\physicalconstants@declare{proton-tau mass ratio}{0.528051}{}{3.6e-05}{0.528051(36)}
\physicalconstants@declare{quantum of circulation}{0.00036369475516}{\square\metre\per\second}{1.1e-13}{0.00036369475516(11)}
\physicalconstants@declare{quantum of circulation times 2}{0.00072738951032}{\square\metre\per\second}{2.2e-13}{0.00072738951032(22)}
\physicalconstants@declare{Rydberg constant}{10973731.56816}{\per\metre}{2.1e-05}{10973731.56816(3)}
\physicalconstants@declare{Rydberg constant times c in Hz}{3289841960250800.0}{\hertz}{6400.0}{3289841960250800.0(64000)}
\physicalconstants@declare{Rydberg constant times hc in eV}{13.605693122994}{\electronvolt}{2.6e-11}{13.605693122994(26)}
\physicalconstants@declare{Rydberg constant times hc in J}{2.1798723611035e-18}{\joule}{4.2e-30}{2.1798723611035(42)e-18}
\physicalconstants@declare{Sackur-Tetrode constant (1 K, 100 kPa)}{-1.15170753706}{}{4.5e-10}{-1.15170753706(45)}
\physicalconstants@declare{Sackur-Tetrode constant (1 K, 101.325 kPa)}{-1.16487052358}{}{4.5e-10}{-1.16487052358(45)}
\physicalconstants@declare{second radiation constant}{0.01438776877}{\metre\kelvin}{0.0}{0.01438776877(0)}
\physicalconstants@declare{shielded helion gyromag. ratio}{203789456.9}{\per\second\per\tesla}{2.4}{203789456.9(24)}
\physicalconstants@declare{shielded helion gyromag. ratio over 2 pi}{32.43409966}{\mega\hertz\per\tesla}{4.3e-07}{32.43409966(43)}
\physicalconstants@declare{shielded helion mag. mom.}{-1.07455309e-26}{\joule\per\tesla}{1.3e-34}{-1.07455309(2)e-26}
\physicalconstants@declare{shielded helion mag. mom. to Bohr magneton ratio}{-0.001158671471}{}{1.4e-11}{-0.001158671471(14)}
\physicalconstants@declare{shielded helion mag. mom. to nuclear magneton ratio}{-2.127497719}{}{2.5e-08}{-2.127497719(25)}
\physicalconstants@declare{shielded helion to proton mag. mom. ratio}{-0.7617665618}{}{8.9e-09}{-0.7617665618(89)}
\physicalconstants@declare{shielded helion to shielded proton mag. mom. ratio}{-0.7617861313}{}{3.3e-09}{-0.7617861313(33)}
\physicalconstants@declare{shielded proton gyromag. ratio}{267515315.1}{\per\second\per\tesla}{2.9}{267515315.1(29)}
\physicalconstants@declare{shielded proton gyromag. ratio over 2 pi}{42.57638507}{\mega\hertz\per\tesla}{5.3e-07}{42.57638507(53)}
\physicalconstants@declare{shielded proton mag. mom.}{1.41057056e-26}{\joule\per\tesla}{1.5e-34}{1.41057056(2)e-26}
\physicalconstants@declare{shielded proton mag. mom. to Bohr magneton ratio}{0.001520993128}{}{1.7e-11}{0.001520993128(17)}
\physicalconstants@declare{shielded proton mag. mom. to nuclear magneton ratio}{2.792775599}{}{3e-08}{2.792775599(30)}
\physicalconstants@declare{speed of light in vacuum}{299792458.0}{\metre\per\second}{0.0}{299792458.0(0)}
\physicalconstants@declare{standard acceleration of gravity}{9.80665}{\metre\per\square\second}{0.0}{9.80665(0)}
\physicalconstants@declare{standard atmosphere}{101325.0}{\pascal}{0.0}{101325.0(0)}
\physicalconstants@declare{Stefan-Boltzmann constant}{5.670374419e-08}{\watt\per\square\metre\per\raiseto{4}\kelvin}{0.0}{5.670374419(0)e-08}
\physicalconstants@declare{tau Compton wavelength}{6.97771e-16}{\metre}{4.7e-20}{6.97771(47)e-16}
\physicalconstants@declare{tau Compton wavelength over 2 pi}{1.11056e-16}{\metre}{1e-20}{1.11056(10)e-16}
\physicalconstants@declare{tau-electron mass ratio}{3477.23}{}{0.23}{3477.23(023)}
\physicalconstants@declare{tau mass}{3.16754e-27}{\kilogram}{2.1e-31}{3.16754(21)e-27}
\physicalconstants@declare{tau mass energy equivalent}{2.84684e-10}{\joule}{1.9e-14}{2.84684(19)e-10}
\physicalconstants@declare{tau mass energy equivalent in MeV}{1776.82}{\mega\electronvolt}{0.16}{1776.82(016)}
\physicalconstants@declare{tau mass in u}{1.90754}{\atomicmassunit}{0.00013}{1.90754(000013)}
\physicalconstants@declare{tau molar mass}{0.00190754}{\kilogram\per\mole}{1.3e-07}{0.00190754(13)}
\physicalconstants@declare{tau-muon mass ratio}{16.817}{}{0.0011}{16.817(2)}
\physicalconstants@declare{tau-neutron mass ratio}{1.89115}{}{0.00013}{1.89115(000013)}
\physicalconstants@declare{tau-proton mass ratio}{1.89376}{}{0.00013}{1.89376(000013)}
\physicalconstants@declare{Thomson cross section}{6.6524587321e-29}{\square\metre}{6e-38}{6.6524587321(60)e-29}
\physicalconstants@declare{triton-electron mag. mom. ratio}{-0.001620514423}{}{2.1e-11}{-0.001620514423(21)}
\physicalconstants@declare{triton-electron mass ratio}{5496.92153573}{}{2.7e-07}{5496.92153573(27)}
\physicalconstants@declare{triton g factor}{5.957924931}{}{1.2e-08}{5.957924931(12)}
\physicalconstants@declare{triton mag. mom.}{1.5046095202e-26}{\joule\per\tesla}{3e-35}{1.5046095202(30)e-26}
\physicalconstants@declare{triton mag. mom. to Bohr magneton ratio}{0.0016223936651}{}{3.2e-12}{0.0016223936651(32)}
\physicalconstants@declare{triton mag. mom. to nuclear magneton ratio}{2.9789624656}{}{5.9e-09}{2.9789624656(59)}
\physicalconstants@declare{triton mass}{5.0073567446e-27}{\kilogram}{1.5e-36}{5.0073567446(15)e-27}
\physicalconstants@declare{triton mass energy equivalent}{4.500387806e-10}{\joule}{1.4e-19}{4.500387806(2)e-10}
\physicalconstants@declare{triton mass energy equivalent in MeV}{2808.92113298}{\mega\electronvolt}{8.5e-07}{2808.92113298(85)}
\physicalconstants@declare{triton mass in u}{3.01550071621}{\atomicmassunit}{1.2e-10}{3.01550071621(12)}
\physicalconstants@declare{triton molar mass}{0.00301550071517}{\kilogram\per\mole}{9.2e-13}{0.00301550071517(92)}
\physicalconstants@declare{triton-neutron mag. mom. ratio}{-1.55718553}{}{3.7e-07}{-1.55718553(37)}
\physicalconstants@declare{triton-proton mag. mom. ratio}{1.066639908}{}{1e-08}{1.066639908(10)}
\physicalconstants@declare{triton-proton mass ratio}{2.99371703414}{}{1.5e-10}{2.99371703414(15)}
\physicalconstants@declare{unified atomic mass unit}{1.6605390666e-27}{\kilogram}{5e-37}{1.6605390666(5)e-27}
\physicalconstants@declare{von Klitzing constant}{25812.80745}{\ohm}{0.0}{25812.80745(0)}
\physicalconstants@declare{weak mixing angle}{0.2229}{}{0.0003}{0.2229(00003)}
\physicalconstants@declare{Wien frequency displacement law constant}{58789257570.0}{\hertz\per\kelvin}{0.0}{58789257570.0(0)}
\physicalconstants@declare{Wien wavelength displacement law constant}{0.002897771955}{\metre\kelvin}{0.0}{0.002897771955(0)}
\physicalconstants@declare{atomic unit of mom.um}{1.992851882e-24}{\kilogram\metre\per\second}{2.4e-32}{1.992851882(24)e-24}
\physicalconstants@declare{electron-helion mass ratio}{0.0001819543074573}{}{7.9e-15}{0.0001819543074573(79)}
\physicalconstants@declare{electron-triton mass ratio}{0.0001819200062251}{}{9e-15}{0.0001819200062251(90)}
\physicalconstants@declare{helion g factor}{-4.255250615}{}{5e-08}{-4.255250615(50)}
\physicalconstants@declare{helion mag. mom.}{-1.074617532e-26}{\joule\per\tesla}{1.3e-34}{-1.074617532(13)e-26}
\physicalconstants@declare{helion mag. mom. to Bohr magneton ratio}{-0.001158740958}{}{1.4e-11}{-0.001158740958(14)}
\physicalconstants@declare{helion mag. mom. to nuclear magneton ratio}{-2.127625307}{}{2.5e-08}{-2.127625307(25)}
\physicalconstants@declare{Loschmidt constant (273.15 K, 100 kPa)}{2.651645804e+25}{\per\cube\metre}{0.0}{2.651645804(0)e+25}
\physicalconstants@declare{natural unit of mom.um}{2.730924488e-22}{\kilogram\metre\per\second}{3.4e-30}{2.730924488(34)e-22}
\physicalconstants@declare{natural unit of mom.um in MeV/c}{0.5109989461}{\mega\electronvolt\clight}{3.1e-09}{0.5109989461(31)}
\physicalconstants@declare{neutron-proton mass difference}{2.30557435e-30}{\kilogram}{8.2e-37}{2.30557435(82)e-30}
\physicalconstants@declare{neutron-proton mass difference energy equivalent}{2.07214689e-13}{\joule}{7.4e-20}{2.07214689(74)e-13}
\physicalconstants@declare{neutron-proton mass difference energy equivalent in MeV}{1.29333236}{\mega\electronvolt}{4.6e-07}{1.29333236(46)}
\physicalconstants@declare{neutron-proton mass difference in u}{0.00138844933}{\atomicmassunit}{4.9e-10}{0.00138844933(49)}
\physicalconstants@declare{standard-state pressure}{100000.0}{\pascal}{0.0}{100000.0(0)}
\physicalconstants@declare{alpha particle relative atomic mass}{4.001506179127}{}{6.3e-11}{4.001506179127(63)}
\physicalconstants@declare{Bohr magneton in inverse meter per tesla}{46.686447783}{\per\metre\per\tesla}{1.4e-08}{46.686447783(14)}
\physicalconstants@declare{Boltzmann constant in inverse meter per kelvin}{69.50348004}{\per\metre\per\kelvin}{0.0}{69.50348004(0)}
\physicalconstants@declare{conventional value of ampere-90}{1.00000008887}{\ampere}{0.0}{1.00000008887(0)}
\physicalconstants@declare{conventional value of coulomb-90}{1.00000008887}{\coulomb}{0.0}{1.00000008887(0)}
\physicalconstants@declare{conventional value of farad-90}{0.9999999822}{\farad}{0.0}{0.9999999822(0)}

  Tipo book, Cantó, ingresé, volá, on tesis umsa %\pgls{MO}

  %  \bibliographystyle{alpha}
  %  \bibliography{biblio} %without extension .bib %--Obsoleto en biblatex
  % \printbibliography %Prints bibliography

  % \begin{alignat*}{5}
  %     &\text{(a)}\qquad &\lfloor n/2\rfloor &<{} &p &\leq n                  &\implies &l_{p}(n\wr)=1 \\
  %     &\text{(b)}\qquad &\lfloor n/3\rfloor &<{} &p &\leq \lfloor n/2\rfloor &\implies &l_{p}(n\wr)=0 \\
  %     &\text{(c)}\qquad &\sqrt{n}           &<{} &p &\leq \lfloor n/3\rfloor &\implies &l_{p}(n\wr)=\lfloor n/p\rfloor \bmod 2\\
  %     &\text{(d)}\qquad &2                  &<{} &p &\leq \sqrt{n}           &\implies &l_{p}(n\wr)<\log_2(n)\\
  %     &\text{(e)}\qquad &                   &    &p &= 2                     &\implies &l_{p}(n\wr)=\sigma_{2}(\lfloor n/2\rfloor)
  % \end{alignat*}
\end{document}



% https://tex.meta.stackexchange.com/questions/228/ive-just-been-asked-to-write-a-minimal-working-example-mwe-what-is-that/3225#3225
% https://tex.stackexchange.com/questions/147502/how-can-i-include-subfiles-in-to-a-subfiles
% https://github.com/github/gitignore/blob/main/TeX.gitignore
% https://github.com/tecosaur/LaTeX-Utilities/wiki/Live-Snippets


% Bibliografia
% http://www.khirevich.com/latex/biblatex/
% https://tex.stackexchange.com/questions/553/what-packages-do-people-load-by-default-in-latex/2974#2974
% https://tex.stackexchange.com/questions/5091/what-to-do-to-switch-to-biblatex
% https://tex.stackexchange.com/questions/25701/bibtex-vs-biber-and-biblatex-vs-natbib

% SVG images
% https://www.baeldung.com/cs/latex-svg-images